%%%%%%%%%%%%%%%%%%%%%%%%%%%%%%%%%%%%%%%%%%%%%%%%%%%%%%%%%%%%%%%%%%%%%%%%%%%%%%%%%%%
%% THE MOTTAZ PREAMBLE %%%%%%%%%%%%%%%%%%%%%%%%%%%%%%%%%%%%%%%%%%%%%%%%%%%%%%%%%%%%
%%%%%%%%%%%%%%%%%%%%%%%%%%%%%%%%%%%%%%%%%%%%%%%%%%%%%%%%%%%%%%%%%%%%%%%%%%%%%%%%%%%

%% Created by: Tony Mottaz
%% Description: This is my customized preamble with which I write all of my homework 
%% assignments.
%% It is ever-expanding to fit new and nitpicky uses.
%% Please refer to the README for change log, goals for the future, and other info.

%% Version: 1.2
%% Publish Date: November 10, 2015
%% Publish Location: Tony's Copy.com cloud storage account


%%%%%%%%%%%%%%%%%%%%%%%%%%%%%%%%%%%%%%%%%%%%%%%%%%%%%%%%%%%%%%%%%%%%%%%%%%%%%%%%%%%
%% SET THE LAYOUT OF THE PAGE %%%%%%%%%%%%%%%%%%%%%%%%%%%%%%%%%%%%%%%%%%%%%%%%%%%%%
%%%%%%%%%%%%%%%%%%%%%%%%%%%%%%%%%%%%%%%%%%%%%%%%%%%%%%%%%%%%%%%%%%%%%%%%%%%%%%%%%%%
% NOTE: This layout assumes that the user is using the letterpaper size.
% Specify this by inserting the keyval option 'letterpaper', e.g.
% '\documentclass[letterpaper]{article}'
% at the beginning of your .tex document.

\usepackage{layout}
\hoffset 0.25in
\voffset 0pt
\oddsidemargin 0pt
\topmargin 0pt
\headheight 1.3em
\headsep 25pt
\textheight 600pt
\textwidth 6in
\marginparwidth 85pt



%%%%%%%%%%%%%%%%%%%%%%%%%%%%%%%%%%%%%%%%%%%%%%%%%%%%%%%%%%%%%%%%%%%%%%%%%%%%%%%%%%%
%% CHOOSE AN ALTERNATIVE FONT %%%%%%%%%%%%%%%%%%%%%%%%%%%%%%%%%%%%%%%%%%%%%%%%%%%%%
%%%%%%%%%%%%%%%%%%%%%%%%%%%%%%%%%%%%%%%%%%%%%%%%%%%%%%%%%%%%%%%%%%%%%%%%%%%%%%%%%%%

\newcommand{\concfont}{%		The classic font created by Donald Knuth
\usepackage{concmath}
\usepackage[T1]{fontenc}
}

\newcommand{\stixfont}{%		Similar to Times, good for textbook-style text
\usepackage[T1]{fontenc}
\usepackage{stix}
}


%%%%%%%%%%%%%%%%%%%%%%%%%%%%%%%%%%%%%%%%%%%%%%%%%%%%%%%%%%%%%%%%%%%%%%%%%%%%%%%%%%%
%% PACKAGES %%%%%%%%%%%%%%%%%%%%%%%%%%%%%%%%%%%%%%%%%%%%%%%%%%%%%%%%%%%%%%%%%%%%%%%
%%%%%%%%%%%%%%%%%%%%%%%%%%%%%%%%%%%%%%%%%%%%%%%%%%%%%%%%%%%%%%%%%%%%%%%%%%%%%%%%%%%

% For basic math typesetting needs
\usepackage{mathtools,amsthm,amssymb}

% For double stroke fonts (\mathds{•}) and math script fonts (\mathscr{•})
\usepackage{dsfont,mathrsfs}

% Math package to show cancellation
\usepackage{cancel}

% For custom enumerations (refer to documentation)
\usepackage{enumerate}

% Can change size of math (refer to documentation)
\usepackage{relsize}

% The Tikz package is utilized for creating mathematical images, and PGFPlots 
% provides tools for creating graphs
\usepackage{tikz,pgfplots}
% This gives PGFPlots backwards compatibility
\pgfplotsset{compat=newest}

% To correct proof environment spacing (see below)
\usepackage{needspace}

% For todo notes in the margin, plus other features (refer to the documentation)
\usepackage[textwidth=0.9\marginparwidth]{todonotes}

% Creates the fancy header
\usepackage{fancyhdr}

% New paragraphs skip a line rather than indent
\usepackage{parskip}

% Creates hyperlinks
\usepackage[colorlinks=true,linkcolor=blue,linkbordercolor=white]{hyperref}

% Additional hyperlink possibilities
\usepackage{cleveref}

% Extended coloring options (refer to documentation)
\usepackage{xcolor}

% Framed environment puts a box around its contents (refer to documentation)
\usepackage{framed}

% For some miscellaneous symbols
\usepackage{wasysym}

% For adding dummy text
\usepackage{lipsum}

% Utilized in footnote symbols definition (see below)
\usepackage{alphalph}

% For inserting PDF documents
\usepackage{pdfpages}



%%%%%%%%%%%%%%%%%%%%%%%%%%%%%%%%%%%%%%%%%%%%%%%%%%%%%%%%%%%%%%%%%%%%%%%%%%%%%%%%%%%
%% USER DEFINED FOOTNOTE SYMBOLS %%%%%%%%%%%%%%%%%%%%%%%%%%%%%%%%%%%%%%%%%%%%%%%%%%
%%%%%%%%%%%%%%%%%%%%%%%%%%%%%%%%%%%%%%%%%%%%%%%%%%%%%%%%%%%%%%%%%%%%%%%%%%%%%%%%%%%
% NOTE: To add another symbol, add '\or\symbol' to the string below,
% as demonstrated.

\makeatletter
\newcommand*{\myfnsymbolsingle}[1]{%
  \ensuremath{%
    \ifcase#1
    \or
      \dag 
    \or
      \ddag
    \or
      \kreuz
    \or
      \star
    \else
      \@ctrerr  
    \fi
  }%   
}   
\makeatother

\newcommand*{\myfnsymbol}[1]{%
  \myfnsymbolsingle{\value{#1}}%
}

% remove upper boundary counting error 
% by multiplying the symbols, if needed

\newalphalph{\myfnsymbolmult}[mult]{\myfnsymbolsingle}{}

\renewcommand*{\thefootnote}{%
  \myfnsymbolmult{\value{footnote}}%
}


%%%%%%%%%%%%%%%%%%%%%%%%%%%%%%%%%%%%%%%%%%%%%%%%%%%%%%%%%%%%%%%%%%%%%%%%%%%%%%%%%%%
%% USER DEFINED COMMANDS & MATH OPERATORS %%%%%%%%%%%%%%%%%%%%%%%%%%%%%%%%%%%%%%%%%
%%%%%%%%%%%%%%%%%%%%%%%%%%%%%%%%%%%%%%%%%%%%%%%%%%%%%%%%%%%%%%%%%%%%%%%%%%%%%%%%%%%

% Shortcuts for double stroke characters
\newcommand{\N}{\mathds{N}}
\newcommand{\Z}{\mathds{Z}}
\newcommand{\Q}{\mathds{Q}}
\newcommand{\R}{\mathds{R}}
\newcommand{\C}{\mathds{C}}
\newcommand{\F}{\mathds{F}}

%% Various mathematical delimeters and reassignment of operators %%

% Places the argument inside auto-scaled angular brackets
\newcommand{\ang}[1]{\left\langle #1 \right\rangle}

% Places the argument inside auto-scaled vertical bars
\newcommand{\abs}[1]{\left\vert #1 \right\vert}

% Places the argument inside auto-scaled double vertical bars
\newcommand{\norm}[1]{\left\| #1 \right\|}

\newcommand{\iso}{\cong}	% The 'is isomorphic to' symbol
\newcommand{\nsg}{\unlhd}	% The 'normal subgroup' symbol
\newcommand{\rnsg}{\unrhd}	% Reversed 'normal subgroup' symbol
\newcommand{\nnsg}{\ntrianglelefteq}	% Negated 'normal subgroup' symbol
\newcommand{\del}{\nabla} 	% The 'del', or 'gradient' operator
\newcommand{\dsum}{\displaystyle\sum\limits}	% Like \dfrac, but for inline sums
\renewcommand{\limsup}{\overline{\lim}\,}	% The 'limit superior' symbol
\renewcommand{\liminf}{\underline{\lim}\,}	% The 'limit inferior' symbol

% Upper integral
\def\upint{\mathchoice
    {\mkern13mu\overline{\vphantom{\intop}\mkern7mu}\mkern-20mu}
    {\mkern7mu\overline{\vphantom{\intop}\mkern7mu}\mkern-14mu}
    {\mkern7mu\overline{\vphantom{\intop}\mkern7mu}\mkern-14mu}
    {\mkern7mu\overline{\vphantom{\intop}\mkern7mu}\mkern-14mu}
  \int}
% Lower integral
\def\lowint{\mkern3mu\underline{\vphantom{\intop}\mkern7mu}\mkern-10mu\int}
\newcommand{\eref}[1]{(\ref{#1})}

% User defined fraction-like objects.
% The genfrac command takes 6 arguments.
% The first two arguments are optional left and right delimiters, respectively
% The third argument is an optional specification of the line thickness
% The fourth argument optionally overrides the mathstyle sizing:
% 0--displaystyle, 1--textstyle, 2--scriptstyle, 3--scriptscriptstyle
% The last two arguments are the numerator and denominator, respectively.
% For example, a binomial coefficient would be written in this environment:
% \newcommand{\binom}[2]{\genfrac{(}{)}{0pt}{}{#1}{#2}}

% User defined math operators
\DeclareMathOperator{\ord}{ord}		% Order
\DeclareMathOperator{\sgn}{sgn}		% Sign function
\DeclareMathOperator{\lcm}{lcm}		% Least common multiple
\DeclareMathOperator{\aut}{Aut}		% Group of automorphisms
\DeclareMathOperator{\inn}{Inn}		% Group of inner automorphisms
\DeclareMathOperator{\sym}{Sym}		% Symmetric group
\DeclareMathOperator{\id}{id}		% Identity operator
\DeclareMathOperator{\img}{Im}		% Image of a function
\DeclareMathOperator{\stab}{Stab}	% Stabilizer
\DeclareMathOperator{\orb}{Orb}		% Orbit
\DeclareMathOperator{\cl}{C\ell}	% Conjugacy class
\DeclareMathOperator{\core}{core}	% Core
\DeclareMathOperator{\syl}{Syl}		% Sylow group
\DeclareMathOperator{\cha}{char}	% Characteristic
\DeclareMathOperator{\tr}{tr}		% Trace of a matrix

% I want math inside of the \textbf{•} command to also be bold
\DeclareTextFontCommand{\textbf}{\boldmath\bfseries}

% For referencing multiple items to single footnote,
% use 'text\footnote{\label{foo}Some text here}', and
% later: 'other text\cref{foo}'
\crefformat{footnote}{#2\footnotemark[#1]#3}


%%%%%%%%%%%%%%%%%%%%%%%%%%%%%%%%%%%%%%%%%%%%%%%%%%%%%%%%%%%%%%%%%%%%%%%%%%%%%%%%%%%
%% THEOREM STYLES %%%%%%%%%%%%%%%%%%%%%%%%%%%%%%%%%%%%%%%%%%%%%%%%%%%%%%%%%%%%%%%%%
%%%%%%%%%%%%%%%%%%%%%%%%%%%%%%%%%%%%%%%%%%%%%%%%%%%%%%%%%%%%%%%%%%%%%%%%%%%%%%%%%%%

% REFER TO the amsthm documentation for further explanation.

% First define the styles.
\newtheoremstyle{first_style} 		% name
    {\topsep}                    	% Space above
    {\topsep}                    	% Space below
    {\slshape}                		% Body font
    {}                           	% Indent amount
    {\bfseries}                  	% Theorem head font
    {.}              				% Punctuation after theorem head
    {.5em}                       	% Space after theorem head
    {}  						  	% Theorem head spec (can be left empty, meaning *normal*)
    
\newtheoremstyle{second_style} 		% name
    {0.5ex}                    		% Space above
    {\topsep}                    	% Space below
    {\normalfont}                	% Body font
    {}                           	% Indent amount
    {\bfseries}                  	% Theorem head font
    {:}              				% Punctuation after theorem head
    {.5em}                       	% Space after theorem head
    {}  						  	% Theorem head spec (can be left empty, meaning *normal*)
    
\newtheoremstyle{third_style} 		% name
    {1ex}	                    	% Space above
    {0pt}   	                 	% Space below
    {\normalfont}                	% Body font
    {}                           	% Indent amount
    {\slshape \bfseries}            % Theorem head font
    {$\,\rightarrow \,$}           	% Punctuation after theorem head
    {0pt}                       	% Space after theorem head
    {}  						 	% Theorem head spec (can be left empty, meaning *normal*)
    
% Now define the theorem environments based on these styles.
\theoremstyle{first_style}
\newtheorem*{thm}{Theorem}
\newtheorem*{lem}{Lemma}
\newtheorem*{prop}{Proposition}
\newtheorem*{cor}{Corollary}

\theoremstyle{second_style}
\newtheorem*{defn}{Definition}
\newtheorem*{exmp}{Example}
\newtheorem*{sol}{Solution}

\theoremstyle{third_style}
\newtheorem*{note}{Note}
\newtheorem*{claim}{Claim}


%%%%%%%%%%%%%%%%%%%%%%%%%%%%%%%%%%%%%%%%%%%%%%%%%%%%%%%%%%%%%%%%%%%%%%%%%%%%%%%%%%%
%% PROOF ENVIRONMENT SPACING %%%%%%%%%%%%%%%%%%%%%%%%%%%%%%%%%%%%%%%%%%%%%%%%%%%%%%
%%%%%%%%%%%%%%%%%%%%%%%%%%%%%%%%%%%%%%%%%%%%%%%%%%%%%%%%%%%%%%%%%%%%%%%%%%%%%%%%%%%

% I prefer a proof to start on a new page if 3 lines or fewer of the proof will
% only be seen on a single page. The 'needspace' package does this for me
% quite nicely.

% First, copy '\proof' and '\endproof' to avoid infinite loop errors:
\expandafter\let\expandafter\oldproof\csname\string\proof\endcsname
\let\oldendproof\endproof

% Now we can redefine the proof environment:
\renewenvironment{proof}[1][\proofname]{%
	\needspace{4\baselineskip} \oldproof[#1]}
	{\oldendproof}


%%%%%%%%%%%%%%%%%%%%%%%%%%%%%%%%%%%%%%%%%%%%%%%%%%%%%%%%%%%%%%%%%%%%%%%%%%%%%%%%%%%
%% DEFINE HEADER %%%%%%%%%%%%%%%%%%%%%%%%%%%%%%%%%%%%%%%%%%%%%%%%%%%%%%%%%%%%%%%%%%
%%%%%%%%%%%%%%%%%%%%%%%%%%%%%%%%%%%%%%%%%%%%%%%%%%%%%%%%%%%%%%%%%%%%%%%%%%%%%%%%%%%
\lhead{<Course Title>}
\chead{<Assignment Title>}
\rhead{Tony Mottaz}		% Replace with your name
\pagestyle{fancy}
