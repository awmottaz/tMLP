%%%%%%%%%%%%%%%%%%%%%%%%%%%%%%%%%%%%%%%%%%%%%%%%%%%%%%%%%%%%%%%%%%%%%%%%%%%%%%%%
%%%%%%%%%%%%%%%%%%%%%%%% the MOTTAZ LaTeX PREAMBLE (NOTES) %%%%%%%%%%%%%%%%%%%%%
%%%%%%%%%%%%%%%%%%%%%%%%%%%%%%%%%%%% v. 1.00 %%%%%%%%%%%%%%%%%%%%%%%%%%%%%%%%%%%
%
%    Author: Anthony Mottaz
%    Version: 1.00  (2016/04/03)
%    Description: The MOTTAZ LaTeX PREAMBLES are a collection of preambles that
%                 have been optimally designed for tasks that are of primary
%                 concern to students and professors of mathematics.
%                    This particular preamble, the 'NOTES' edition, has lots of
%                 preloaded packages and features that are useful for typing
%                 course notes for a mathematics class. For a complete list and
%                 descriptions of these features, refer to the Usage Guide PDF
%                 document. For release notes and version change logs, see
%                 'notes_changelog.log'.
%                    Lots of time and thought has gone into creating the most
%                 comprehensive and useful preamble possible, but the work is
%                 never done. If you experience issues with this preamble or
%                 have any suggestions for improvement, please contact me at
%
%                                 anthonywmottaz@gmail.com
%
%                 Happy TeXing!
%                 - A.M.
%
%    HOW TO USE: At the beginning of your document, after '\documentclass[•]{•}'
%                add '%%%%%%%%%%%%%%%%%%%%%%%%%%%%%%%%%%%%%%%%%%%%%%%%%%%%%%%%%%%%%%%%%%%%%%%%%%%%%%%%
%%%%%%%%%%%%%%%%%%%%%%%% the MOTTAZ LaTeX PREAMBLE (NOTES) %%%%%%%%%%%%%%%%%%%%%
%%%%%%%%%%%%%%%%%%%%%%%%%%%%%%%%%%%% v. 1.00 %%%%%%%%%%%%%%%%%%%%%%%%%%%%%%%%%%%
%
%    Author: Anthony Mottaz
%    Version: 1.00  (2016/04/03)
%    Description: The MOTTAZ LaTeX PREAMBLES are a collection of preambles that
%                 have been optimally designed for tasks that are of primary
%                 concern to students and professors of mathematics.
%                    This particular preamble, the 'NOTES' edition, has lots of
%                 preloaded packages and features that are useful for typing
%                 course notes for a mathematics class. For a complete list and
%                 descriptions of these features, refer to the Usage Guide PDF
%                 document. For release notes and version change logs, see
%                 'notes_changelog.log'.
%                    Lots of time and thought has gone into creating the most
%                 comprehensive and useful preamble possible, but the work is
%                 never done. If you experience issues with this preamble or
%                 have any suggestions for improvement, please contact me at
%
%                                 anthonywmottaz@gmail.com
%
%                 Happy TeXing!
%                 - A.M.
%
%    HOW TO USE: At the beginning of your document, after '\documentclass[•]{•}'
%                add '%%%%%%%%%%%%%%%%%%%%%%%%%%%%%%%%%%%%%%%%%%%%%%%%%%%%%%%%%%%%%%%%%%%%%%%%%%%%%%%%
%%%%%%%%%%%%%%%%%%%%%%%% the MOTTAZ LaTeX PREAMBLE (NOTES) %%%%%%%%%%%%%%%%%%%%%
%%%%%%%%%%%%%%%%%%%%%%%%%%%%%%%%%%%% v. 1.00 %%%%%%%%%%%%%%%%%%%%%%%%%%%%%%%%%%%
%
%    Author: Anthony Mottaz
%    Version: 1.00  (2016/04/03)
%    Description: The MOTTAZ LaTeX PREAMBLES are a collection of preambles that
%                 have been optimally designed for tasks that are of primary
%                 concern to students and professors of mathematics.
%                    This particular preamble, the 'NOTES' edition, has lots of
%                 preloaded packages and features that are useful for typing
%                 course notes for a mathematics class. For a complete list and
%                 descriptions of these features, refer to the Usage Guide PDF
%                 document. For release notes and version change logs, see
%                 'notes_changelog.log'.
%                    Lots of time and thought has gone into creating the most
%                 comprehensive and useful preamble possible, but the work is
%                 never done. If you experience issues with this preamble or
%                 have any suggestions for improvement, please contact me at
%
%                                 anthonywmottaz@gmail.com
%
%                 Happy TeXing!
%                 - A.M.
%
%    HOW TO USE: At the beginning of your document, after '\documentclass[•]{•}'
%                add '%%%%%%%%%%%%%%%%%%%%%%%%%%%%%%%%%%%%%%%%%%%%%%%%%%%%%%%%%%%%%%%%%%%%%%%%%%%%%%%%
%%%%%%%%%%%%%%%%%%%%%%%% the MOTTAZ LaTeX PREAMBLE (NOTES) %%%%%%%%%%%%%%%%%%%%%
%%%%%%%%%%%%%%%%%%%%%%%%%%%%%%%%%%%% v. 1.00 %%%%%%%%%%%%%%%%%%%%%%%%%%%%%%%%%%%
%
%    Author: Anthony Mottaz
%    Version: 1.00  (2016/04/03)
%    Description: The MOTTAZ LaTeX PREAMBLES are a collection of preambles that
%                 have been optimally designed for tasks that are of primary
%                 concern to students and professors of mathematics.
%                    This particular preamble, the 'NOTES' edition, has lots of
%                 preloaded packages and features that are useful for typing
%                 course notes for a mathematics class. For a complete list and
%                 descriptions of these features, refer to the Usage Guide PDF
%                 document. For release notes and version change logs, see
%                 'notes_changelog.log'.
%                    Lots of time and thought has gone into creating the most
%                 comprehensive and useful preamble possible, but the work is
%                 never done. If you experience issues with this preamble or
%                 have any suggestions for improvement, please contact me at
%
%                                 anthonywmottaz@gmail.com
%
%                 Happy TeXing!
%                 - A.M.
%
%    HOW TO USE: At the beginning of your document, after '\documentclass[•]{•}'
%                add '\input{path/to/notes_preamble.tex}' where
%                'path/to/notes_preamble.tex'
%                is the location of the 'notes_preamble.tex' file on your
%                system.
%
%    LICENSE:    This project is released under the LaTeX Project Public
%                License, v1.3c or later. See
%
%                          http://www.latex-project.org/lppl.txt
%
% The following is for use with 'mylatexformat'
%\documentclass[letterpaper]{article}
%
%-------------------------------------------------------------------------------
%             PAGE LAYOUT                                                      |
%-------------------------------------------------------------------------------
%
\usepackage{layout}
%
%   The following section defines the page layout for the document. Notice that
%   every defined length is based on either the size of the page or the size of 
%   the font. This is so that any choice in paper size and font results in a
%   similar layout. To understand the motivation behind these dimension choices,
%   refer to the Usage Guide.
%
\usepackage{geometry}
\geometry{
    headheight = 1.3em,
    headsep = 1.3em,
    marginparsep = 1.5em,
    footnotesep = 3em,
    hdivide={0.15\paperwidth,0.62\paperwidth,*},
    vdivide={0.1\paperheight,0.78\paperheight,*},
    marginparwidth=0.18\paperwidth
}
%
%-------------------------------------------------------------------------------
%             LOAD ALL OF THE PACKAGES                                         |
%-------------------------------------------------------------------------------
%
%   This is where we load all of the packages that may be useful in creating
%   your document. Refer to the Usage Guide for thorough information on the
%   application of each package.
%
%   WARNING: Do not change the order in which these packages are loaded. Doing
%            so may result in errors.
%
\usepackage{lmodern}
\usepackage[T1]{fontenc}
\usepackage[utf8]{inputenc}
\usepackage{mathtools}
\usepackage{amsthm}
\usepackage{amssymb}
\usepackage{dsfont}
\usepackage{mathrsfs}
\usepackage{cancel}
\usepackage[shortlabels]{enumitem}
\usepackage{array}
    \setlength{\extrarowheight}{1pt}
\usepackage{arydshln}
\usepackage{relsize}
\usepackage[dvipsnames]{xcolor}
\usepackage{tikz}
    \usetikzlibrary{shapes}
    \usetikzlibrary{arrows}
\usepackage{pgfplots}
    \pgfplotsset{compat = 1.10}% You may use a later version, if you wish.
    \usepgfplotslibrary{fillbetween}
\usepackage{needspace}
\usepackage[textwidth = 0.9\marginparwidth]{todonotes}
\usepackage{fancyhdr}
\usepackage[parfill]{parskip}
\usepackage{imakeidx}
\usepackage[
    colorlinks=true,
    linkcolor=blue,
    linkbordercolor=white,
    urlcolor=blue,
    unicode
    ]{hyperref}
\usepackage{cleveref}
\usepackage{framed}
\usepackage{wasysym}
\usepackage{lipsum}
\usepackage{alphalph}
\usepackage{pdfpages}
\usepackage{float}
\usepackage{tabularx}
\usepackage{textgreek}
\usepackage{upgreek}
\usepackage{signchart}
\usepackage{microtype}
\usepackage{multicol}
\usepackage{tikz-cd}
\usepackage{fancyvrb}
%
%-------------------------------------------------------------------------------
%             FOOTNOTE SYMBOLS                                                 |
%-------------------------------------------------------------------------------
%
%   Since the intended use of this preamble is for the creating of mathematical
%   documents, we must be careful that the footnote symbols will not be confused
%   with mathematical symbols, e.g. a '2' footnote mark being mistaken as the
%   square of a number. This section redefines the footnote symbols. To add your
%   own symbol to the list, simply append
%     \or
%       \newsymbol
%   to the list below.
%
\makeatletter%
\newcommand*{\myfnsymbolsingle}[1]{%
  \ensuremath{%
    \ifcase#1
    \or
      \dag
    \or
      \ddag
    \or
      \kreuz
    \or
      \star
    \else
      \@ctrerr  
    \fi
  }%   
}   
\makeatother%
\newcommand*{\myfnsymbol}[1]{%
  \myfnsymbolsingle{\value{#1}}%
}
\newalphalph{\myfnsymbolmult}[mult]{\myfnsymbolsingle}{}%
\renewcommand*{\thefootnote}{%
  \myfnsymbolmult{\value{footnote}}%
}
%
% For referencing multiple locations to single footnote, use
%   '\footnote{\label{foo}<text in the footnote>}',
%     and later use: '\cref{foo}'
%
\crefformat{footnote}{#2\footnotemark[#1]#3}%
%
%-------------------------------------------------------------------------------
%             MATH-SPECIFIC COMMANDS, OPERATORS, AND DEFINITIONS               |
%-------------------------------------------------------------------------------
%
%----- DOUBLE STROKE CHARACTERS -----%
\newcommand{\N}{\mathds{N}}      % Naturals
\newcommand{\Z}{\mathds{Z}}      % Integers
\newcommand{\Q}{\mathds{Q}}      % Rationals
\newcommand{\R}{\mathds{R}}      % Reals
\renewcommand{\C}{\mathds{C}}    % Complex numbers
\renewcommand{\H}{\mathds{H}}    % Quaternions (Hamiltonions)
\newcommand{\F}{\mathds{F}}      % Generic field
%
%----- AUTOMATICALLY SCALED DELIMITERS -----%
\DeclarePairedDelimiter{\paren}{(}{)}%                                  ( )
\DeclarePairedDelimiter{\ang}{\langle}{\rangle}%                        < >
\DeclarePairedDelimiter{\brc}{\{}{\}}%                                  { }
\DeclarePairedDelimiter{\brkt}{[}{]}%                                   [ ]
\DeclarePairedDelimiter{\abs}{\lvert}{\rvert}%                          | |
\DeclarePairedDelimiter{\norm}{\lVert}{\rVert}%                        || ||
\renewcommand{\mid}{\mkern4mu\middle\vert\mkern4mu}
%
% Swap the definition of starred and non-starred versions.
% Non-starred delimeters will auto-scale, and starred versions will not.
\makeatletter
% \paren
\let\oldparen\paren
\def\paren{\@ifstar{\oldparen}{\oldparen*}}
% \ang
\let\oldang\ang
\def\ang{\@ifstar{\oldang}{\oldang*}}
% \brc
\let\oldbrc\brc
\def\brc{\@ifstar{\oldbrc}{\oldbrc*}}
% \brkt
\let\oldbrkt\brkt
\def\brkt{\@ifstar{\oldbrkt}{\oldbrkt*}}
% \abs
\let\oldabs\abs
\def\abs{\@ifstar{\oldabs}{\oldabs*}}
% \norm
\let\oldnorm\norm
\def\norm{\@ifstar{\oldnorm}{\oldnorm*}}
\makeatother
%
%----- FUNCTIONS -----%
\newcommand{\ceil}[1]{\left\lceil #1 \right\rceil}      % Ceiling function
\newcommand{\floor}[1]{\left\lfloor #1 \right\rfloor}   % Floor function
\renewcommand{\bar}[1]{                                 % The complex conjugate
    \mkern 1mu\overline{\mkern-1mu#1\mkern-1mu}}
%
%----- OPERATORS -----%
\newcommand{\iso}{\cong}                   % The 'is isomorphic to' symbol
\newcommand{\nsg}{\unlhd}                  % The 'normal subgroup' symbol
\newcommand{\rnsg}{\unrhd}                 % Reversed 'normal subgroup' symbol
\newcommand{\nnsg}{\ntrianglelefteq}       % Negated 'normal subgroup' symbol
\newcommand{\del}{\nabla}                  % The 'del', or 'gradient' operator
\DeclareMathOperator{\ord}{ord}            % Order
\DeclareMathOperator{\sgn}{sgn}            % Sign function
\DeclareMathOperator{\lcm}{lcm}            % Least common multiple
\DeclareMathOperator{\aut}{Aut}            % Group of automorphisms
\DeclareMathOperator{\inn}{Inn}            % Group of inner automorphisms
\DeclareMathOperator{\sym}{Sym}            % Symmetric group
\DeclareMathOperator{\id}{id}              % Identity operator
\DeclareMathOperator{\img}{Im}             % Image of a function
\DeclareMathOperator{\stab}{Stab}          % Stabilizer
\DeclareMathOperator{\orb}{Orb}            % Orbit
\DeclareMathOperator{\cl}{C\ell}           % Conjugacy class
\DeclareMathOperator{\core}{core}          % Core
\DeclareMathOperator{\syl}{Syl}            % Sylow group
\DeclareMathOperator{\cha}{char}           % Characteristic
\DeclareMathOperator{\tr}{tr}              % Trace of a matrix
\DeclareMathOperator{\fun}{Fun}            % Set of functions
\DeclareMathOperator{\cis}{cis}            % Cos + i Sin
\DeclareMathOperator{\Arg}{Arg}            % Principal argument
\DeclareMathOperator{\Frac}{Frac}          % Field of fractions
\DeclareMathOperator{\ann}{Ann}            % Annihilator
\DeclareMathOperator{\tor}{Tor}            % Torsion set
\DeclareMathOperator{\End}{End}            % Set of endomorphisms
\DeclareMathOperator{\Hom}{Hom}            % Set of homomorphisms
\renewcommand{\limsup}{\varlimsup}         % Limit superior
\renewcommand{\liminf}{\varliminf}         % Limit inferior
\renewcommand{\Re}{\operatorname{Re}}      % Real part
\renewcommand{\Im}{\operatorname{Im}}      % Imaginary part
% Upper integral
\def\upint{\mathchoice
    {\mkern13mu\overline{\vphantom{\intop}\mkern7mu}\mkern-20mu}
    {\mkern7mu\overline{\vphantom{\intop}\mkern7mu}\mkern-14mu}
    {\mkern7mu\overline{\vphantom{\intop}\mkern7mu}\mkern-14mu}
    {\mkern7mu\overline{\vphantom{\intop}\mkern7mu}\mkern-14mu}
    \int}
% Lower integral
\def\lowint{\mkern3mu\underline{\vphantom{\intop}\mkern7mu}\mkern-10mu\int}
\newcommand{\dsum}{\displaystyle\sum\limits}% Like \dfrac, but for inline sums
% Big 'boxplus' symbol
\newcommand{\bigboxplus}{%
  \mathop{%
    \mathchoice{\dobigboxplus\Large}%
               {\dobigboxplus\large}
               {\dobigboxplus\normalsize}
               {\dobigboxplus\small}
    }\displaylimits
}
%
\newcommand{\dobigboxplus}[1]{%
\vcenter{#1\kern.2ex\hbox{$\boxplus$}\kern.2ex}}
% 'Divides' relation
\newcommand{\divides}{\bigm|}
\newcommand{\ndivides}{%
  \mathrel{\mkern.5mu % small adjustment
    % superimpose \nmid to \big|
    \ooalign{\hidewidth$\big|$\hidewidth\cr$\nmid$\cr}%
  }%
}
%
%----- UNCATEGORIZED -----%
%   I want math inside of \textbf{•} to also be bolded font
\DeclareTextFontCommand{\textbf}{\boldmath\bfseries}
%
%   You can create your own fraction-like commands using '\genfrac'. This takes
%   6 inputs:
%      1. left delimiter
%      2. right delimiter
%      3. fraction bar line thickness
%      4. mathstyle size override:
%         0--displaystyle, 1--textstyle, 2--scriptstyle, 3--scriptscriptstyle
%      5. numerator content
%      6. denominator content
%
%-------------------------------------------------------------------------------
%             THEOREM STYLES                                                   |
%-------------------------------------------------------------------------------
%
%   Here we define some custom styles for theorem-like environments. Refer to
%   the 'amsthm' documentation for an explanation of how this is done.
%
%----- DEFINE THE STYLES -----%
\newtheoremstyle{first_style}
    {\baselineskip}
    {0.7\baselineskip}
    {\slshape}
    {}
    {\bfseries}
    {.}
    {0.5em}
    {}
\newtheoremstyle{second_style}
    {\baselineskip}
    {0.7\baselineskip}
    {\normalfont}
    {}
    {\bfseries}
    {:}
    {0.5em}
    {}
\newtheoremstyle{third_style}
    {1ex}
    {-0.5\baselineskip}
    {\normalfont}
    {}
    {\slshape\bfseries}
    {:}
    {0.25em}
    {}
%----- DEFINE THE ENVIRONMENTS -----%
\theoremstyle{first_style}
    \newtheorem{nthm}{Theorem}[section]
    \newtheorem{nlem}[nthm]{Lemma}
    \newtheorem{nprop}[nthm]{Proposition}
    \newtheorem{ncor}[nthm]{Corollary}
    \newtheorem*{thm}{Theorem}
    \newtheorem*{lem}{Lemma}
    \newtheorem*{prop}{Proposition}
    \newtheorem*{cor}{Corollary}
\theoremstyle{second_style}
    \newtheorem*{defn}{Definition}
    \newtheorem*{exmp}{Example}
    \newtheorem*{sol}{Solution}
    \newtheorem*{case}{Case}
\theoremstyle{third_style}
    \newtheorem*{note}{Note}
    \newtheorem*{claim}{Claim}
%----- OPTIONAL FRAMED VERSIONS -----%
\newcommand{\frameddefinitions}{
    \let\olddefn\defn
    \let\oldenddefn\enddefn
    \renewenvironment{defn}{\begin{framed} \olddefn}{\oldenddefn \end{framed}}
}
\newcommand{\framedtheorems}{
    \let\oldthm\thm
    \let\oldendthm\endthm
    \renewenvironment{thm}{\begin{framed} \oldthm}{\oldendthm \end{framed}}
    \let\oldlem\lem
    \let\oldendlem\endlem
    \renewenvironment{lem}{\begin{framed} \oldlem}{\oldendlem \end{framed}}
    \let\oldprop\prop
    \let\oldendprop\endprop
    \renewenvironment{prop}{\begin{framed} \oldprop}{\oldendprop \end{framed}}
    \let\oldcor\cor
    \let\oldendcor\endcor
    \renewenvironment{cor}{\begin{framed} \oldcor}{\oldendcor \end{framed}}
}
\newcommand{\framedntheorems}{
    \let\oldnthm\nthm
    \let\oldendnthm\endnthm
    \renewenvironment{nthm}{\begin{framed} \oldnthm}{\oldendnthm \end{framed}}
    \let\oldnlem\nlem
    \let\oldendnlem\endnlem
    \renewenvironment{nlem}{\begin{framed} \oldnlem}{\oldendnlem \end{framed}}
    \let\oldnprop\nprop
    \let\oldendnprop\endnprop
    \renewenvironment{nprop}{\begin{framed} \oldnprop}{\oldendnprop
    \end{framed}}
    \let\oldncor\ncor
    \let\oldendncor\endncor
    \renewenvironment{ncor}{\begin{framed} \oldncor}{\oldendncor \end{framed}}
}
%
%-------------------------------------------------------------------------------
%             THE HEADER                                                       |
%-------------------------------------------------------------------------------
%
%   We use the 'fancyhdr' package to create our header. Here I place the <Course
%   Title> on the left, <Homework Title> in the center, and <Your Name> on the
%   right. Since your name will not change between documents, I recommend
%   specifying your name here like I did.
%
\newcommand{\cn}{<Course Title>}
\newcommand{\as}{<Assignment Title>}
\newcommand{\nm}{Tony Mottaz}       % <== REPLACE WITH YOUR NAME
\newcommand{\coursetitle}[1]{\renewcommand{\cn}{#1}}
\newcommand{\hwtitle}[1]{\renewcommand{\as}{#1}}
\newcommand{\myname}[1]{\renewcommand{\nm}{#1}}
\lhead{\scshape \cn}
\chead{\scshape \as}
\rhead{\scshape \nm}
\pagestyle{fancy}
%
%-------------------------------------------------------------------------------
%             UNCATEGORIZED FEATURES                                           |
%-------------------------------------------------------------------------------
%
%----- PROOF ENVIRONMENT -----%
%
%   Here I redefine the proof environment so that if fewer than 4 lines will
%   show at the bottom of a page, then the entire environment will be pushed
%   onto the next page.
\expandafter\let\expandafter\oldproof\csname\string\proof\endcsname
\let\oldendproof\endproof
\renewenvironment{proof}[1][\proofname]{%
    \Needspace*{4\baselineskip} \oldproof[#1]}
    {\oldendproof}
%
%----- DARK THEME -----%
%
%   Here I provide the command '\darktheme' to turn your output PDF completely
%   dark. This lowers eye strain when editing documents for long time periods.
\newcommand{\darktheme}{
    \pagecolor[rgb]{0.1,0.1,0.1}
    \color[rgb]{0.95,0.95,0.95}
}
%
%----- CIRCLED -----%
%
%   If you want something circled, this will do it nicely.
\newcommand*\circled[1]{\tikz[baseline=(char.base)]{%
    \node[shape=ellipse,draw,inner sep=1pt] (char) {#1};}}
%
%----- MARGIN ALIGN -----%
%
%   If you want item labels aligned at the margins, use
%   \begin{enumerate}[align=margin,labelsep=0pt]
\SetLabelAlign{margin}{\llap{#1~~}}
%
%----- MARGIN PARAGRAPHS -----%
\newcommand{\mpar}[1]{\marginpar{\footnotesize#1}}
%
% The following is for use with 'mylatexformat'
%\csname endofdump\endcsname
' where
%                'path/to/notes_preamble.tex'
%                is the location of the 'notes_preamble.tex' file on your
%                system.
%
%    LICENSE:    This project is released under the LaTeX Project Public
%                License, v1.3c or later. See
%
%                          http://www.latex-project.org/lppl.txt
%
% The following is for use with 'mylatexformat'
%\documentclass[letterpaper]{article}
%
%-------------------------------------------------------------------------------
%             PAGE LAYOUT                                                      |
%-------------------------------------------------------------------------------
%
\usepackage{layout}
%
%   The following section defines the page layout for the document. Notice that
%   every defined length is based on either the size of the page or the size of 
%   the font. This is so that any choice in paper size and font results in a
%   similar layout. To understand the motivation behind these dimension choices,
%   refer to the Usage Guide.
%
\usepackage{geometry}
\geometry{
    headheight = 1.3em,
    headsep = 1.3em,
    marginparsep = 1.5em,
    footnotesep = 3em,
    hdivide={0.15\paperwidth,0.62\paperwidth,*},
    vdivide={0.1\paperheight,0.78\paperheight,*},
    marginparwidth=0.18\paperwidth
}
%
%-------------------------------------------------------------------------------
%             LOAD ALL OF THE PACKAGES                                         |
%-------------------------------------------------------------------------------
%
%   This is where we load all of the packages that may be useful in creating
%   your document. Refer to the Usage Guide for thorough information on the
%   application of each package.
%
%   WARNING: Do not change the order in which these packages are loaded. Doing
%            so may result in errors.
%
\usepackage{lmodern}
\usepackage[T1]{fontenc}
\usepackage[utf8]{inputenc}
\usepackage{mathtools}
\usepackage{amsthm}
\usepackage{amssymb}
\usepackage{dsfont}
\usepackage{mathrsfs}
\usepackage{cancel}
\usepackage[shortlabels]{enumitem}
\usepackage{array}
    \setlength{\extrarowheight}{1pt}
\usepackage{arydshln}
\usepackage{relsize}
\usepackage[dvipsnames]{xcolor}
\usepackage{tikz}
    \usetikzlibrary{shapes}
    \usetikzlibrary{arrows}
\usepackage{pgfplots}
    \pgfplotsset{compat = 1.10}% You may use a later version, if you wish.
    \usepgfplotslibrary{fillbetween}
\usepackage{needspace}
\usepackage[textwidth = 0.9\marginparwidth]{todonotes}
\usepackage{fancyhdr}
\usepackage[parfill]{parskip}
\usepackage{imakeidx}
\usepackage[
    colorlinks=true,
    linkcolor=blue,
    linkbordercolor=white,
    urlcolor=blue,
    unicode
    ]{hyperref}
\usepackage{cleveref}
\usepackage{framed}
\usepackage{wasysym}
\usepackage{lipsum}
\usepackage{alphalph}
\usepackage{pdfpages}
\usepackage{float}
\usepackage{tabularx}
\usepackage{textgreek}
\usepackage{upgreek}
\usepackage{signchart}
\usepackage{microtype}
\usepackage{multicol}
\usepackage{tikz-cd}
\usepackage{fancyvrb}
%
%-------------------------------------------------------------------------------
%             FOOTNOTE SYMBOLS                                                 |
%-------------------------------------------------------------------------------
%
%   Since the intended use of this preamble is for the creating of mathematical
%   documents, we must be careful that the footnote symbols will not be confused
%   with mathematical symbols, e.g. a '2' footnote mark being mistaken as the
%   square of a number. This section redefines the footnote symbols. To add your
%   own symbol to the list, simply append
%     \or
%       \newsymbol
%   to the list below.
%
\makeatletter%
\newcommand*{\myfnsymbolsingle}[1]{%
  \ensuremath{%
    \ifcase#1
    \or
      \dag
    \or
      \ddag
    \or
      \kreuz
    \or
      \star
    \else
      \@ctrerr  
    \fi
  }%   
}   
\makeatother%
\newcommand*{\myfnsymbol}[1]{%
  \myfnsymbolsingle{\value{#1}}%
}
\newalphalph{\myfnsymbolmult}[mult]{\myfnsymbolsingle}{}%
\renewcommand*{\thefootnote}{%
  \myfnsymbolmult{\value{footnote}}%
}
%
% For referencing multiple locations to single footnote, use
%   '\footnote{\label{foo}<text in the footnote>}',
%     and later use: '\cref{foo}'
%
\crefformat{footnote}{#2\footnotemark[#1]#3}%
%
%-------------------------------------------------------------------------------
%             MATH-SPECIFIC COMMANDS, OPERATORS, AND DEFINITIONS               |
%-------------------------------------------------------------------------------
%
%----- DOUBLE STROKE CHARACTERS -----%
\newcommand{\N}{\mathds{N}}      % Naturals
\newcommand{\Z}{\mathds{Z}}      % Integers
\newcommand{\Q}{\mathds{Q}}      % Rationals
\newcommand{\R}{\mathds{R}}      % Reals
\renewcommand{\C}{\mathds{C}}    % Complex numbers
\renewcommand{\H}{\mathds{H}}    % Quaternions (Hamiltonions)
\newcommand{\F}{\mathds{F}}      % Generic field
%
%----- AUTOMATICALLY SCALED DELIMITERS -----%
\DeclarePairedDelimiter{\paren}{(}{)}%                                  ( )
\DeclarePairedDelimiter{\ang}{\langle}{\rangle}%                        < >
\DeclarePairedDelimiter{\brc}{\{}{\}}%                                  { }
\DeclarePairedDelimiter{\brkt}{[}{]}%                                   [ ]
\DeclarePairedDelimiter{\abs}{\lvert}{\rvert}%                          | |
\DeclarePairedDelimiter{\norm}{\lVert}{\rVert}%                        || ||
\renewcommand{\mid}{\mkern4mu\middle\vert\mkern4mu}
%
% Swap the definition of starred and non-starred versions.
% Non-starred delimeters will auto-scale, and starred versions will not.
\makeatletter
% \paren
\let\oldparen\paren
\def\paren{\@ifstar{\oldparen}{\oldparen*}}
% \ang
\let\oldang\ang
\def\ang{\@ifstar{\oldang}{\oldang*}}
% \brc
\let\oldbrc\brc
\def\brc{\@ifstar{\oldbrc}{\oldbrc*}}
% \brkt
\let\oldbrkt\brkt
\def\brkt{\@ifstar{\oldbrkt}{\oldbrkt*}}
% \abs
\let\oldabs\abs
\def\abs{\@ifstar{\oldabs}{\oldabs*}}
% \norm
\let\oldnorm\norm
\def\norm{\@ifstar{\oldnorm}{\oldnorm*}}
\makeatother
%
%----- FUNCTIONS -----%
\newcommand{\ceil}[1]{\left\lceil #1 \right\rceil}      % Ceiling function
\newcommand{\floor}[1]{\left\lfloor #1 \right\rfloor}   % Floor function
\renewcommand{\bar}[1]{                                 % The complex conjugate
    \mkern 1mu\overline{\mkern-1mu#1\mkern-1mu}}
%
%----- OPERATORS -----%
\newcommand{\iso}{\cong}                   % The 'is isomorphic to' symbol
\newcommand{\nsg}{\unlhd}                  % The 'normal subgroup' symbol
\newcommand{\rnsg}{\unrhd}                 % Reversed 'normal subgroup' symbol
\newcommand{\nnsg}{\ntrianglelefteq}       % Negated 'normal subgroup' symbol
\newcommand{\del}{\nabla}                  % The 'del', or 'gradient' operator
\DeclareMathOperator{\ord}{ord}            % Order
\DeclareMathOperator{\sgn}{sgn}            % Sign function
\DeclareMathOperator{\lcm}{lcm}            % Least common multiple
\DeclareMathOperator{\aut}{Aut}            % Group of automorphisms
\DeclareMathOperator{\inn}{Inn}            % Group of inner automorphisms
\DeclareMathOperator{\sym}{Sym}            % Symmetric group
\DeclareMathOperator{\id}{id}              % Identity operator
\DeclareMathOperator{\img}{Im}             % Image of a function
\DeclareMathOperator{\stab}{Stab}          % Stabilizer
\DeclareMathOperator{\orb}{Orb}            % Orbit
\DeclareMathOperator{\cl}{C\ell}           % Conjugacy class
\DeclareMathOperator{\core}{core}          % Core
\DeclareMathOperator{\syl}{Syl}            % Sylow group
\DeclareMathOperator{\cha}{char}           % Characteristic
\DeclareMathOperator{\tr}{tr}              % Trace of a matrix
\DeclareMathOperator{\fun}{Fun}            % Set of functions
\DeclareMathOperator{\cis}{cis}            % Cos + i Sin
\DeclareMathOperator{\Arg}{Arg}            % Principal argument
\DeclareMathOperator{\Frac}{Frac}          % Field of fractions
\DeclareMathOperator{\ann}{Ann}            % Annihilator
\DeclareMathOperator{\tor}{Tor}            % Torsion set
\DeclareMathOperator{\End}{End}            % Set of endomorphisms
\DeclareMathOperator{\Hom}{Hom}            % Set of homomorphisms
\renewcommand{\limsup}{\varlimsup}         % Limit superior
\renewcommand{\liminf}{\varliminf}         % Limit inferior
\renewcommand{\Re}{\operatorname{Re}}      % Real part
\renewcommand{\Im}{\operatorname{Im}}      % Imaginary part
% Upper integral
\def\upint{\mathchoice
    {\mkern13mu\overline{\vphantom{\intop}\mkern7mu}\mkern-20mu}
    {\mkern7mu\overline{\vphantom{\intop}\mkern7mu}\mkern-14mu}
    {\mkern7mu\overline{\vphantom{\intop}\mkern7mu}\mkern-14mu}
    {\mkern7mu\overline{\vphantom{\intop}\mkern7mu}\mkern-14mu}
    \int}
% Lower integral
\def\lowint{\mkern3mu\underline{\vphantom{\intop}\mkern7mu}\mkern-10mu\int}
\newcommand{\dsum}{\displaystyle\sum\limits}% Like \dfrac, but for inline sums
% Big 'boxplus' symbol
\newcommand{\bigboxplus}{%
  \mathop{%
    \mathchoice{\dobigboxplus\Large}%
               {\dobigboxplus\large}
               {\dobigboxplus\normalsize}
               {\dobigboxplus\small}
    }\displaylimits
}
%
\newcommand{\dobigboxplus}[1]{%
\vcenter{#1\kern.2ex\hbox{$\boxplus$}\kern.2ex}}
% 'Divides' relation
\newcommand{\divides}{\bigm|}
\newcommand{\ndivides}{%
  \mathrel{\mkern.5mu % small adjustment
    % superimpose \nmid to \big|
    \ooalign{\hidewidth$\big|$\hidewidth\cr$\nmid$\cr}%
  }%
}
%
%----- UNCATEGORIZED -----%
%   I want math inside of \textbf{•} to also be bolded font
\DeclareTextFontCommand{\textbf}{\boldmath\bfseries}
%
%   You can create your own fraction-like commands using '\genfrac'. This takes
%   6 inputs:
%      1. left delimiter
%      2. right delimiter
%      3. fraction bar line thickness
%      4. mathstyle size override:
%         0--displaystyle, 1--textstyle, 2--scriptstyle, 3--scriptscriptstyle
%      5. numerator content
%      6. denominator content
%
%-------------------------------------------------------------------------------
%             THEOREM STYLES                                                   |
%-------------------------------------------------------------------------------
%
%   Here we define some custom styles for theorem-like environments. Refer to
%   the 'amsthm' documentation for an explanation of how this is done.
%
%----- DEFINE THE STYLES -----%
\newtheoremstyle{first_style}
    {\baselineskip}
    {0.7\baselineskip}
    {\slshape}
    {}
    {\bfseries}
    {.}
    {0.5em}
    {}
\newtheoremstyle{second_style}
    {\baselineskip}
    {0.7\baselineskip}
    {\normalfont}
    {}
    {\bfseries}
    {:}
    {0.5em}
    {}
\newtheoremstyle{third_style}
    {1ex}
    {-0.5\baselineskip}
    {\normalfont}
    {}
    {\slshape\bfseries}
    {:}
    {0.25em}
    {}
%----- DEFINE THE ENVIRONMENTS -----%
\theoremstyle{first_style}
    \newtheorem{nthm}{Theorem}[section]
    \newtheorem{nlem}[nthm]{Lemma}
    \newtheorem{nprop}[nthm]{Proposition}
    \newtheorem{ncor}[nthm]{Corollary}
    \newtheorem*{thm}{Theorem}
    \newtheorem*{lem}{Lemma}
    \newtheorem*{prop}{Proposition}
    \newtheorem*{cor}{Corollary}
\theoremstyle{second_style}
    \newtheorem*{defn}{Definition}
    \newtheorem*{exmp}{Example}
    \newtheorem*{sol}{Solution}
    \newtheorem*{case}{Case}
\theoremstyle{third_style}
    \newtheorem*{note}{Note}
    \newtheorem*{claim}{Claim}
%----- OPTIONAL FRAMED VERSIONS -----%
\newcommand{\frameddefinitions}{
    \let\olddefn\defn
    \let\oldenddefn\enddefn
    \renewenvironment{defn}{\begin{framed} \olddefn}{\oldenddefn \end{framed}}
}
\newcommand{\framedtheorems}{
    \let\oldthm\thm
    \let\oldendthm\endthm
    \renewenvironment{thm}{\begin{framed} \oldthm}{\oldendthm \end{framed}}
    \let\oldlem\lem
    \let\oldendlem\endlem
    \renewenvironment{lem}{\begin{framed} \oldlem}{\oldendlem \end{framed}}
    \let\oldprop\prop
    \let\oldendprop\endprop
    \renewenvironment{prop}{\begin{framed} \oldprop}{\oldendprop \end{framed}}
    \let\oldcor\cor
    \let\oldendcor\endcor
    \renewenvironment{cor}{\begin{framed} \oldcor}{\oldendcor \end{framed}}
}
\newcommand{\framedntheorems}{
    \let\oldnthm\nthm
    \let\oldendnthm\endnthm
    \renewenvironment{nthm}{\begin{framed} \oldnthm}{\oldendnthm \end{framed}}
    \let\oldnlem\nlem
    \let\oldendnlem\endnlem
    \renewenvironment{nlem}{\begin{framed} \oldnlem}{\oldendnlem \end{framed}}
    \let\oldnprop\nprop
    \let\oldendnprop\endnprop
    \renewenvironment{nprop}{\begin{framed} \oldnprop}{\oldendnprop
    \end{framed}}
    \let\oldncor\ncor
    \let\oldendncor\endncor
    \renewenvironment{ncor}{\begin{framed} \oldncor}{\oldendncor \end{framed}}
}
%
%-------------------------------------------------------------------------------
%             THE HEADER                                                       |
%-------------------------------------------------------------------------------
%
%   We use the 'fancyhdr' package to create our header. Here I place the <Course
%   Title> on the left, <Homework Title> in the center, and <Your Name> on the
%   right. Since your name will not change between documents, I recommend
%   specifying your name here like I did.
%
\newcommand{\cn}{<Course Title>}
\newcommand{\as}{<Assignment Title>}
\newcommand{\nm}{Tony Mottaz}       % <== REPLACE WITH YOUR NAME
\newcommand{\coursetitle}[1]{\renewcommand{\cn}{#1}}
\newcommand{\hwtitle}[1]{\renewcommand{\as}{#1}}
\newcommand{\myname}[1]{\renewcommand{\nm}{#1}}
\lhead{\scshape \cn}
\chead{\scshape \as}
\rhead{\scshape \nm}
\pagestyle{fancy}
%
%-------------------------------------------------------------------------------
%             UNCATEGORIZED FEATURES                                           |
%-------------------------------------------------------------------------------
%
%----- PROOF ENVIRONMENT -----%
%
%   Here I redefine the proof environment so that if fewer than 4 lines will
%   show at the bottom of a page, then the entire environment will be pushed
%   onto the next page.
\expandafter\let\expandafter\oldproof\csname\string\proof\endcsname
\let\oldendproof\endproof
\renewenvironment{proof}[1][\proofname]{%
    \Needspace*{4\baselineskip} \oldproof[#1]}
    {\oldendproof}
%
%----- DARK THEME -----%
%
%   Here I provide the command '\darktheme' to turn your output PDF completely
%   dark. This lowers eye strain when editing documents for long time periods.
\newcommand{\darktheme}{
    \pagecolor[rgb]{0.1,0.1,0.1}
    \color[rgb]{0.95,0.95,0.95}
}
%
%----- CIRCLED -----%
%
%   If you want something circled, this will do it nicely.
\newcommand*\circled[1]{\tikz[baseline=(char.base)]{%
    \node[shape=ellipse,draw,inner sep=1pt] (char) {#1};}}
%
%----- MARGIN ALIGN -----%
%
%   If you want item labels aligned at the margins, use
%   \begin{enumerate}[align=margin,labelsep=0pt]
\SetLabelAlign{margin}{\llap{#1~~}}
%
%----- MARGIN PARAGRAPHS -----%
\newcommand{\mpar}[1]{\marginpar{\footnotesize#1}}
%
% The following is for use with 'mylatexformat'
%\csname endofdump\endcsname
' where
%                'path/to/notes_preamble.tex'
%                is the location of the 'notes_preamble.tex' file on your
%                system.
%
%    LICENSE:    This project is released under the LaTeX Project Public
%                License, v1.3c or later. See
%
%                          http://www.latex-project.org/lppl.txt
%
% The following is for use with 'mylatexformat'
%\documentclass[letterpaper]{article}
%
%-------------------------------------------------------------------------------
%             PAGE LAYOUT                                                      |
%-------------------------------------------------------------------------------
%
\usepackage{layout}
%
%   The following section defines the page layout for the document. Notice that
%   every defined length is based on either the size of the page or the size of 
%   the font. This is so that any choice in paper size and font results in a
%   similar layout. To understand the motivation behind these dimension choices,
%   refer to the Usage Guide.
%
\usepackage{geometry}
\geometry{
    headheight = 1.3em,
    headsep = 1.3em,
    marginparsep = 1.5em,
    footnotesep = 3em,
    hdivide={0.15\paperwidth,0.62\paperwidth,*},
    vdivide={0.1\paperheight,0.78\paperheight,*},
    marginparwidth=0.18\paperwidth
}
%
%-------------------------------------------------------------------------------
%             LOAD ALL OF THE PACKAGES                                         |
%-------------------------------------------------------------------------------
%
%   This is where we load all of the packages that may be useful in creating
%   your document. Refer to the Usage Guide for thorough information on the
%   application of each package.
%
%   WARNING: Do not change the order in which these packages are loaded. Doing
%            so may result in errors.
%
\usepackage{lmodern}
\usepackage[T1]{fontenc}
\usepackage[utf8]{inputenc}
\usepackage{mathtools}
\usepackage{amsthm}
\usepackage{amssymb}
\usepackage{dsfont}
\usepackage{mathrsfs}
\usepackage{cancel}
\usepackage[shortlabels]{enumitem}
\usepackage{array}
    \setlength{\extrarowheight}{1pt}
\usepackage{arydshln}
\usepackage{relsize}
\usepackage[dvipsnames]{xcolor}
\usepackage{tikz}
    \usetikzlibrary{shapes}
    \usetikzlibrary{arrows}
\usepackage{pgfplots}
    \pgfplotsset{compat = 1.10}% You may use a later version, if you wish.
    \usepgfplotslibrary{fillbetween}
\usepackage{needspace}
\usepackage[textwidth = 0.9\marginparwidth]{todonotes}
\usepackage{fancyhdr}
\usepackage[parfill]{parskip}
\usepackage{imakeidx}
\usepackage[
    colorlinks=true,
    linkcolor=blue,
    linkbordercolor=white,
    urlcolor=blue,
    unicode
    ]{hyperref}
\usepackage{cleveref}
\usepackage{framed}
\usepackage{wasysym}
\usepackage{lipsum}
\usepackage{alphalph}
\usepackage{pdfpages}
\usepackage{float}
\usepackage{tabularx}
\usepackage{textgreek}
\usepackage{upgreek}
\usepackage{signchart}
\usepackage{microtype}
\usepackage{multicol}
\usepackage{tikz-cd}
\usepackage{fancyvrb}
%
%-------------------------------------------------------------------------------
%             FOOTNOTE SYMBOLS                                                 |
%-------------------------------------------------------------------------------
%
%   Since the intended use of this preamble is for the creating of mathematical
%   documents, we must be careful that the footnote symbols will not be confused
%   with mathematical symbols, e.g. a '2' footnote mark being mistaken as the
%   square of a number. This section redefines the footnote symbols. To add your
%   own symbol to the list, simply append
%     \or
%       \newsymbol
%   to the list below.
%
\makeatletter%
\newcommand*{\myfnsymbolsingle}[1]{%
  \ensuremath{%
    \ifcase#1
    \or
      \dag
    \or
      \ddag
    \or
      \kreuz
    \or
      \star
    \else
      \@ctrerr  
    \fi
  }%   
}   
\makeatother%
\newcommand*{\myfnsymbol}[1]{%
  \myfnsymbolsingle{\value{#1}}%
}
\newalphalph{\myfnsymbolmult}[mult]{\myfnsymbolsingle}{}%
\renewcommand*{\thefootnote}{%
  \myfnsymbolmult{\value{footnote}}%
}
%
% For referencing multiple locations to single footnote, use
%   '\footnote{\label{foo}<text in the footnote>}',
%     and later use: '\cref{foo}'
%
\crefformat{footnote}{#2\footnotemark[#1]#3}%
%
%-------------------------------------------------------------------------------
%             MATH-SPECIFIC COMMANDS, OPERATORS, AND DEFINITIONS               |
%-------------------------------------------------------------------------------
%
%----- DOUBLE STROKE CHARACTERS -----%
\newcommand{\N}{\mathds{N}}      % Naturals
\newcommand{\Z}{\mathds{Z}}      % Integers
\newcommand{\Q}{\mathds{Q}}      % Rationals
\newcommand{\R}{\mathds{R}}      % Reals
\renewcommand{\C}{\mathds{C}}    % Complex numbers
\renewcommand{\H}{\mathds{H}}    % Quaternions (Hamiltonions)
\newcommand{\F}{\mathds{F}}      % Generic field
%
%----- AUTOMATICALLY SCALED DELIMITERS -----%
\DeclarePairedDelimiter{\paren}{(}{)}%                                  ( )
\DeclarePairedDelimiter{\ang}{\langle}{\rangle}%                        < >
\DeclarePairedDelimiter{\brc}{\{}{\}}%                                  { }
\DeclarePairedDelimiter{\brkt}{[}{]}%                                   [ ]
\DeclarePairedDelimiter{\abs}{\lvert}{\rvert}%                          | |
\DeclarePairedDelimiter{\norm}{\lVert}{\rVert}%                        || ||
\renewcommand{\mid}{\mkern4mu\middle\vert\mkern4mu}
%
% Swap the definition of starred and non-starred versions.
% Non-starred delimeters will auto-scale, and starred versions will not.
\makeatletter
% \paren
\let\oldparen\paren
\def\paren{\@ifstar{\oldparen}{\oldparen*}}
% \ang
\let\oldang\ang
\def\ang{\@ifstar{\oldang}{\oldang*}}
% \brc
\let\oldbrc\brc
\def\brc{\@ifstar{\oldbrc}{\oldbrc*}}
% \brkt
\let\oldbrkt\brkt
\def\brkt{\@ifstar{\oldbrkt}{\oldbrkt*}}
% \abs
\let\oldabs\abs
\def\abs{\@ifstar{\oldabs}{\oldabs*}}
% \norm
\let\oldnorm\norm
\def\norm{\@ifstar{\oldnorm}{\oldnorm*}}
\makeatother
%
%----- FUNCTIONS -----%
\newcommand{\ceil}[1]{\left\lceil #1 \right\rceil}      % Ceiling function
\newcommand{\floor}[1]{\left\lfloor #1 \right\rfloor}   % Floor function
\renewcommand{\bar}[1]{                                 % The complex conjugate
    \mkern 1mu\overline{\mkern-1mu#1\mkern-1mu}}
%
%----- OPERATORS -----%
\newcommand{\iso}{\cong}                   % The 'is isomorphic to' symbol
\newcommand{\nsg}{\unlhd}                  % The 'normal subgroup' symbol
\newcommand{\rnsg}{\unrhd}                 % Reversed 'normal subgroup' symbol
\newcommand{\nnsg}{\ntrianglelefteq}       % Negated 'normal subgroup' symbol
\newcommand{\del}{\nabla}                  % The 'del', or 'gradient' operator
\DeclareMathOperator{\ord}{ord}            % Order
\DeclareMathOperator{\sgn}{sgn}            % Sign function
\DeclareMathOperator{\lcm}{lcm}            % Least common multiple
\DeclareMathOperator{\aut}{Aut}            % Group of automorphisms
\DeclareMathOperator{\inn}{Inn}            % Group of inner automorphisms
\DeclareMathOperator{\sym}{Sym}            % Symmetric group
\DeclareMathOperator{\id}{id}              % Identity operator
\DeclareMathOperator{\img}{Im}             % Image of a function
\DeclareMathOperator{\stab}{Stab}          % Stabilizer
\DeclareMathOperator{\orb}{Orb}            % Orbit
\DeclareMathOperator{\cl}{C\ell}           % Conjugacy class
\DeclareMathOperator{\core}{core}          % Core
\DeclareMathOperator{\syl}{Syl}            % Sylow group
\DeclareMathOperator{\cha}{char}           % Characteristic
\DeclareMathOperator{\tr}{tr}              % Trace of a matrix
\DeclareMathOperator{\fun}{Fun}            % Set of functions
\DeclareMathOperator{\cis}{cis}            % Cos + i Sin
\DeclareMathOperator{\Arg}{Arg}            % Principal argument
\DeclareMathOperator{\Frac}{Frac}          % Field of fractions
\DeclareMathOperator{\ann}{Ann}            % Annihilator
\DeclareMathOperator{\tor}{Tor}            % Torsion set
\DeclareMathOperator{\End}{End}            % Set of endomorphisms
\DeclareMathOperator{\Hom}{Hom}            % Set of homomorphisms
\renewcommand{\limsup}{\varlimsup}         % Limit superior
\renewcommand{\liminf}{\varliminf}         % Limit inferior
\renewcommand{\Re}{\operatorname{Re}}      % Real part
\renewcommand{\Im}{\operatorname{Im}}      % Imaginary part
% Upper integral
\def\upint{\mathchoice
    {\mkern13mu\overline{\vphantom{\intop}\mkern7mu}\mkern-20mu}
    {\mkern7mu\overline{\vphantom{\intop}\mkern7mu}\mkern-14mu}
    {\mkern7mu\overline{\vphantom{\intop}\mkern7mu}\mkern-14mu}
    {\mkern7mu\overline{\vphantom{\intop}\mkern7mu}\mkern-14mu}
    \int}
% Lower integral
\def\lowint{\mkern3mu\underline{\vphantom{\intop}\mkern7mu}\mkern-10mu\int}
\newcommand{\dsum}{\displaystyle\sum\limits}% Like \dfrac, but for inline sums
% Big 'boxplus' symbol
\newcommand{\bigboxplus}{%
  \mathop{%
    \mathchoice{\dobigboxplus\Large}%
               {\dobigboxplus\large}
               {\dobigboxplus\normalsize}
               {\dobigboxplus\small}
    }\displaylimits
}
%
\newcommand{\dobigboxplus}[1]{%
\vcenter{#1\kern.2ex\hbox{$\boxplus$}\kern.2ex}}
% 'Divides' relation
\newcommand{\divides}{\bigm|}
\newcommand{\ndivides}{%
  \mathrel{\mkern.5mu % small adjustment
    % superimpose \nmid to \big|
    \ooalign{\hidewidth$\big|$\hidewidth\cr$\nmid$\cr}%
  }%
}
%
%----- UNCATEGORIZED -----%
%   I want math inside of \textbf{•} to also be bolded font
\DeclareTextFontCommand{\textbf}{\boldmath\bfseries}
%
%   You can create your own fraction-like commands using '\genfrac'. This takes
%   6 inputs:
%      1. left delimiter
%      2. right delimiter
%      3. fraction bar line thickness
%      4. mathstyle size override:
%         0--displaystyle, 1--textstyle, 2--scriptstyle, 3--scriptscriptstyle
%      5. numerator content
%      6. denominator content
%
%-------------------------------------------------------------------------------
%             THEOREM STYLES                                                   |
%-------------------------------------------------------------------------------
%
%   Here we define some custom styles for theorem-like environments. Refer to
%   the 'amsthm' documentation for an explanation of how this is done.
%
%----- DEFINE THE STYLES -----%
\newtheoremstyle{first_style}
    {\baselineskip}
    {0.7\baselineskip}
    {\slshape}
    {}
    {\bfseries}
    {.}
    {0.5em}
    {}
\newtheoremstyle{second_style}
    {\baselineskip}
    {0.7\baselineskip}
    {\normalfont}
    {}
    {\bfseries}
    {:}
    {0.5em}
    {}
\newtheoremstyle{third_style}
    {1ex}
    {-0.5\baselineskip}
    {\normalfont}
    {}
    {\slshape\bfseries}
    {:}
    {0.25em}
    {}
%----- DEFINE THE ENVIRONMENTS -----%
\theoremstyle{first_style}
    \newtheorem{nthm}{Theorem}[section]
    \newtheorem{nlem}[nthm]{Lemma}
    \newtheorem{nprop}[nthm]{Proposition}
    \newtheorem{ncor}[nthm]{Corollary}
    \newtheorem*{thm}{Theorem}
    \newtheorem*{lem}{Lemma}
    \newtheorem*{prop}{Proposition}
    \newtheorem*{cor}{Corollary}
\theoremstyle{second_style}
    \newtheorem*{defn}{Definition}
    \newtheorem*{exmp}{Example}
    \newtheorem*{sol}{Solution}
    \newtheorem*{case}{Case}
\theoremstyle{third_style}
    \newtheorem*{note}{Note}
    \newtheorem*{claim}{Claim}
%----- OPTIONAL FRAMED VERSIONS -----%
\newcommand{\frameddefinitions}{
    \let\olddefn\defn
    \let\oldenddefn\enddefn
    \renewenvironment{defn}{\begin{framed} \olddefn}{\oldenddefn \end{framed}}
}
\newcommand{\framedtheorems}{
    \let\oldthm\thm
    \let\oldendthm\endthm
    \renewenvironment{thm}{\begin{framed} \oldthm}{\oldendthm \end{framed}}
    \let\oldlem\lem
    \let\oldendlem\endlem
    \renewenvironment{lem}{\begin{framed} \oldlem}{\oldendlem \end{framed}}
    \let\oldprop\prop
    \let\oldendprop\endprop
    \renewenvironment{prop}{\begin{framed} \oldprop}{\oldendprop \end{framed}}
    \let\oldcor\cor
    \let\oldendcor\endcor
    \renewenvironment{cor}{\begin{framed} \oldcor}{\oldendcor \end{framed}}
}
\newcommand{\framedntheorems}{
    \let\oldnthm\nthm
    \let\oldendnthm\endnthm
    \renewenvironment{nthm}{\begin{framed} \oldnthm}{\oldendnthm \end{framed}}
    \let\oldnlem\nlem
    \let\oldendnlem\endnlem
    \renewenvironment{nlem}{\begin{framed} \oldnlem}{\oldendnlem \end{framed}}
    \let\oldnprop\nprop
    \let\oldendnprop\endnprop
    \renewenvironment{nprop}{\begin{framed} \oldnprop}{\oldendnprop
    \end{framed}}
    \let\oldncor\ncor
    \let\oldendncor\endncor
    \renewenvironment{ncor}{\begin{framed} \oldncor}{\oldendncor \end{framed}}
}
%
%-------------------------------------------------------------------------------
%             THE HEADER                                                       |
%-------------------------------------------------------------------------------
%
%   We use the 'fancyhdr' package to create our header. Here I place the <Course
%   Title> on the left, <Homework Title> in the center, and <Your Name> on the
%   right. Since your name will not change between documents, I recommend
%   specifying your name here like I did.
%
\newcommand{\cn}{<Course Title>}
\newcommand{\as}{<Assignment Title>}
\newcommand{\nm}{Tony Mottaz}       % <== REPLACE WITH YOUR NAME
\newcommand{\coursetitle}[1]{\renewcommand{\cn}{#1}}
\newcommand{\hwtitle}[1]{\renewcommand{\as}{#1}}
\newcommand{\myname}[1]{\renewcommand{\nm}{#1}}
\lhead{\scshape \cn}
\chead{\scshape \as}
\rhead{\scshape \nm}
\pagestyle{fancy}
%
%-------------------------------------------------------------------------------
%             UNCATEGORIZED FEATURES                                           |
%-------------------------------------------------------------------------------
%
%----- PROOF ENVIRONMENT -----%
%
%   Here I redefine the proof environment so that if fewer than 4 lines will
%   show at the bottom of a page, then the entire environment will be pushed
%   onto the next page.
\expandafter\let\expandafter\oldproof\csname\string\proof\endcsname
\let\oldendproof\endproof
\renewenvironment{proof}[1][\proofname]{%
    \Needspace*{4\baselineskip} \oldproof[#1]}
    {\oldendproof}
%
%----- DARK THEME -----%
%
%   Here I provide the command '\darktheme' to turn your output PDF completely
%   dark. This lowers eye strain when editing documents for long time periods.
\newcommand{\darktheme}{
    \pagecolor[rgb]{0.1,0.1,0.1}
    \color[rgb]{0.95,0.95,0.95}
}
%
%----- CIRCLED -----%
%
%   If you want something circled, this will do it nicely.
\newcommand*\circled[1]{\tikz[baseline=(char.base)]{%
    \node[shape=ellipse,draw,inner sep=1pt] (char) {#1};}}
%
%----- MARGIN ALIGN -----%
%
%   If you want item labels aligned at the margins, use
%   \begin{enumerate}[align=margin,labelsep=0pt]
\SetLabelAlign{margin}{\llap{#1~~}}
%
%----- MARGIN PARAGRAPHS -----%
\newcommand{\mpar}[1]{\marginpar{\footnotesize#1}}
%
% The following is for use with 'mylatexformat'
%\csname endofdump\endcsname
' where
%                'path/to/notes_preamble.tex'
%                is the location of the 'notes_preamble.tex' file on your
%                system.
%
%    LICENSE:    This project is released under the LaTeX Project Public
%                License, v1.3c or later. See
%
%                          http://www.latex-project.org/lppl.txt
%
% The following is for use with 'mylatexformat'
%\documentclass[letterpaper]{article}
%
%-------------------------------------------------------------------------------
%             PAGE LAYOUT                                                      |
%-------------------------------------------------------------------------------
%
\usepackage{layout}
%
%   The following section defines the page layout for the document. Notice that
%   every defined length is based on either the size of the page or the size of 
%   the font. This is so that any choice in paper size and font results in a
%   similar layout. To understand the motivation behind these dimension choices,
%   refer to the Usage Guide.
%
\usepackage{geometry}
\geometry{
    headheight = 1.3em,
    headsep = 1.3em,
    marginparsep = 1.5em,
    footnotesep = 3em,
    hdivide={0.15\paperwidth,0.62\paperwidth,*},
    vdivide={0.1\paperheight,0.78\paperheight,*},
    marginparwidth=0.18\paperwidth
}
%
%-------------------------------------------------------------------------------
%             LOAD ALL OF THE PACKAGES                                         |
%-------------------------------------------------------------------------------
%
%   This is where we load all of the packages that may be useful in creating
%   your document. Refer to the Usage Guide for thorough information on the
%   application of each package.
%
%   WARNING: Do not change the order in which these packages are loaded. Doing
%            so may result in errors.
%
\usepackage{lmodern}
\usepackage[T1]{fontenc}
\usepackage[utf8]{inputenc}
\usepackage{mathtools}
\usepackage{amsthm}
\usepackage{amssymb}
\usepackage{dsfont}
\usepackage{mathrsfs}
\usepackage{cancel}
\usepackage[shortlabels]{enumitem}
\usepackage{array}
    \setlength{\extrarowheight}{1pt}
\usepackage{arydshln}
\usepackage{relsize}
\usepackage[dvipsnames]{xcolor}
\usepackage{tikz}
    \usetikzlibrary{shapes}
    \usetikzlibrary{arrows}
\usepackage{pgfplots}
    \pgfplotsset{compat = 1.10}% You may use a later version, if you wish.
    \usepgfplotslibrary{fillbetween}
\usepackage{needspace}
\usepackage[textwidth = 0.9\marginparwidth]{todonotes}
\usepackage{fancyhdr}
\usepackage[parfill]{parskip}
\usepackage{imakeidx}
\usepackage[
    colorlinks=true,
    linkcolor=blue,
    linkbordercolor=white,
    urlcolor=blue,
    unicode
    ]{hyperref}
\usepackage{cleveref}
\usepackage{framed}
\usepackage{wasysym}
\usepackage{lipsum}
\usepackage{alphalph}
\usepackage{pdfpages}
\usepackage{float}
\usepackage{tabularx}
\usepackage{textgreek}
\usepackage{upgreek}
\usepackage{signchart}
\usepackage{microtype}
\usepackage{multicol}
\usepackage{tikz-cd}
\usepackage{fancyvrb}
%
%-------------------------------------------------------------------------------
%             FOOTNOTE SYMBOLS                                                 |
%-------------------------------------------------------------------------------
%
%   Since the intended use of this preamble is for the creating of mathematical
%   documents, we must be careful that the footnote symbols will not be confused
%   with mathematical symbols, e.g. a '2' footnote mark being mistaken as the
%   square of a number. This section redefines the footnote symbols. To add your
%   own symbol to the list, simply append
%     \or
%       \newsymbol
%   to the list below.
%
\makeatletter%
\newcommand*{\myfnsymbolsingle}[1]{%
  \ensuremath{%
    \ifcase#1
    \or
      \dag
    \or
      \ddag
    \or
      \kreuz
    \or
      \star
    \else
      \@ctrerr  
    \fi
  }%   
}   
\makeatother%
\newcommand*{\myfnsymbol}[1]{%
  \myfnsymbolsingle{\value{#1}}%
}
\newalphalph{\myfnsymbolmult}[mult]{\myfnsymbolsingle}{}%
\renewcommand*{\thefootnote}{%
  \myfnsymbolmult{\value{footnote}}%
}
%
% For referencing multiple locations to single footnote, use
%   '\footnote{\label{foo}<text in the footnote>}',
%     and later use: '\cref{foo}'
%
\crefformat{footnote}{#2\footnotemark[#1]#3}%
%
%-------------------------------------------------------------------------------
%             MATH-SPECIFIC COMMANDS, OPERATORS, AND DEFINITIONS               |
%-------------------------------------------------------------------------------
%
%----- DOUBLE STROKE CHARACTERS -----%
\newcommand{\N}{\mathds{N}}      % Naturals
\newcommand{\Z}{\mathds{Z}}      % Integers
\newcommand{\Q}{\mathds{Q}}      % Rationals
\newcommand{\R}{\mathds{R}}      % Reals
\renewcommand{\C}{\mathds{C}}    % Complex numbers
\renewcommand{\H}{\mathds{H}}    % Quaternions (Hamiltonions)
\newcommand{\F}{\mathds{F}}      % Generic field
%
%----- AUTOMATICALLY SCALED DELIMITERS -----%
\DeclarePairedDelimiter{\paren}{(}{)}%                                  ( )
\DeclarePairedDelimiter{\ang}{\langle}{\rangle}%                        < >
\DeclarePairedDelimiter{\brc}{\{}{\}}%                                  { }
\DeclarePairedDelimiter{\brkt}{[}{]}%                                   [ ]
\DeclarePairedDelimiter{\abs}{\lvert}{\rvert}%                          | |
\DeclarePairedDelimiter{\norm}{\lVert}{\rVert}%                        || ||
\renewcommand{\mid}{\mkern4mu\middle\vert\mkern4mu}
%
% Swap the definition of starred and non-starred versions.
% Non-starred delimeters will auto-scale, and starred versions will not.
\makeatletter
% \paren
\let\oldparen\paren
\def\paren{\@ifstar{\oldparen}{\oldparen*}}
% \ang
\let\oldang\ang
\def\ang{\@ifstar{\oldang}{\oldang*}}
% \brc
\let\oldbrc\brc
\def\brc{\@ifstar{\oldbrc}{\oldbrc*}}
% \brkt
\let\oldbrkt\brkt
\def\brkt{\@ifstar{\oldbrkt}{\oldbrkt*}}
% \abs
\let\oldabs\abs
\def\abs{\@ifstar{\oldabs}{\oldabs*}}
% \norm
\let\oldnorm\norm
\def\norm{\@ifstar{\oldnorm}{\oldnorm*}}
\makeatother
%
%----- FUNCTIONS -----%
\newcommand{\ceil}[1]{\left\lceil #1 \right\rceil}      % Ceiling function
\newcommand{\floor}[1]{\left\lfloor #1 \right\rfloor}   % Floor function
\renewcommand{\bar}[1]{                                 % The complex conjugate
    \mkern 1mu\overline{\mkern-1mu#1\mkern-1mu}}
%
%----- OPERATORS -----%
\newcommand{\iso}{\cong}                   % The 'is isomorphic to' symbol
\newcommand{\nsg}{\unlhd}                  % The 'normal subgroup' symbol
\newcommand{\rnsg}{\unrhd}                 % Reversed 'normal subgroup' symbol
\newcommand{\nnsg}{\ntrianglelefteq}       % Negated 'normal subgroup' symbol
\newcommand{\del}{\nabla}                  % The 'del', or 'gradient' operator
\DeclareMathOperator{\ord}{ord}            % Order
\DeclareMathOperator{\sgn}{sgn}            % Sign function
\DeclareMathOperator{\lcm}{lcm}            % Least common multiple
\DeclareMathOperator{\aut}{Aut}            % Group of automorphisms
\DeclareMathOperator{\inn}{Inn}            % Group of inner automorphisms
\DeclareMathOperator{\sym}{Sym}            % Symmetric group
\DeclareMathOperator{\id}{id}              % Identity operator
\DeclareMathOperator{\img}{Im}             % Image of a function
\DeclareMathOperator{\stab}{Stab}          % Stabilizer
\DeclareMathOperator{\orb}{Orb}            % Orbit
\DeclareMathOperator{\cl}{C\ell}           % Conjugacy class
\DeclareMathOperator{\core}{core}          % Core
\DeclareMathOperator{\syl}{Syl}            % Sylow group
\DeclareMathOperator{\cha}{char}           % Characteristic
\DeclareMathOperator{\tr}{tr}              % Trace of a matrix
\DeclareMathOperator{\fun}{Fun}            % Set of functions
\DeclareMathOperator{\cis}{cis}            % Cos + i Sin
\DeclareMathOperator{\Arg}{Arg}            % Principal argument
\DeclareMathOperator{\Frac}{Frac}          % Field of fractions
\DeclareMathOperator{\ann}{Ann}            % Annihilator
\DeclareMathOperator{\tor}{Tor}            % Torsion set
\DeclareMathOperator{\End}{End}            % Set of endomorphisms
\DeclareMathOperator{\Hom}{Hom}            % Set of homomorphisms
\renewcommand{\limsup}{\varlimsup}         % Limit superior
\renewcommand{\liminf}{\varliminf}         % Limit inferior
\renewcommand{\Re}{\operatorname{Re}}      % Real part
\renewcommand{\Im}{\operatorname{Im}}      % Imaginary part
% Upper integral
\def\upint{\mathchoice
    {\mkern13mu\overline{\vphantom{\intop}\mkern7mu}\mkern-20mu}
    {\mkern7mu\overline{\vphantom{\intop}\mkern7mu}\mkern-14mu}
    {\mkern7mu\overline{\vphantom{\intop}\mkern7mu}\mkern-14mu}
    {\mkern7mu\overline{\vphantom{\intop}\mkern7mu}\mkern-14mu}
    \int}
% Lower integral
\def\lowint{\mkern3mu\underline{\vphantom{\intop}\mkern7mu}\mkern-10mu\int}
\newcommand{\dsum}{\displaystyle\sum\limits}% Like \dfrac, but for inline sums
% Big 'boxplus' symbol
\newcommand{\bigboxplus}{%
  \mathop{%
    \mathchoice{\dobigboxplus\Large}%
               {\dobigboxplus\large}
               {\dobigboxplus\normalsize}
               {\dobigboxplus\small}
    }\displaylimits
}
%
\newcommand{\dobigboxplus}[1]{%
\vcenter{#1\kern.2ex\hbox{$\boxplus$}\kern.2ex}}
% 'Divides' relation
\newcommand{\divides}{\bigm|}
\newcommand{\ndivides}{%
  \mathrel{\mkern.5mu % small adjustment
    % superimpose \nmid to \big|
    \ooalign{\hidewidth$\big|$\hidewidth\cr$\nmid$\cr}%
  }%
}
%
%----- UNCATEGORIZED -----%
%   I want math inside of \textbf{•} to also be bolded font
\DeclareTextFontCommand{\textbf}{\boldmath\bfseries}
%
%   You can create your own fraction-like commands using '\genfrac'. This takes
%   6 inputs:
%      1. left delimiter
%      2. right delimiter
%      3. fraction bar line thickness
%      4. mathstyle size override:
%         0--displaystyle, 1--textstyle, 2--scriptstyle, 3--scriptscriptstyle
%      5. numerator content
%      6. denominator content
%
%-------------------------------------------------------------------------------
%             THEOREM STYLES                                                   |
%-------------------------------------------------------------------------------
%
%   Here we define some custom styles for theorem-like environments. Refer to
%   the 'amsthm' documentation for an explanation of how this is done.
%
%----- DEFINE THE STYLES -----%
\newtheoremstyle{first_style}
    {\baselineskip}
    {0.7\baselineskip}
    {\slshape}
    {}
    {\bfseries}
    {.}
    {0.5em}
    {}
\newtheoremstyle{second_style}
    {\baselineskip}
    {0.7\baselineskip}
    {\normalfont}
    {}
    {\bfseries}
    {:}
    {0.5em}
    {}
\newtheoremstyle{third_style}
    {1ex}
    {-0.5\baselineskip}
    {\normalfont}
    {}
    {\slshape\bfseries}
    {:}
    {0.25em}
    {}
%----- DEFINE THE ENVIRONMENTS -----%
\theoremstyle{first_style}
    \newtheorem{nthm}{Theorem}[section]
    \newtheorem{nlem}[nthm]{Lemma}
    \newtheorem{nprop}[nthm]{Proposition}
    \newtheorem{ncor}[nthm]{Corollary}
    \newtheorem*{thm}{Theorem}
    \newtheorem*{lem}{Lemma}
    \newtheorem*{prop}{Proposition}
    \newtheorem*{cor}{Corollary}
\theoremstyle{second_style}
    \newtheorem*{defn}{Definition}
    \newtheorem*{exmp}{Example}
    \newtheorem*{sol}{Solution}
    \newtheorem*{case}{Case}
\theoremstyle{third_style}
    \newtheorem*{note}{Note}
    \newtheorem*{claim}{Claim}
%----- OPTIONAL FRAMED VERSIONS -----%
\newcommand{\frameddefinitions}{
    \let\olddefn\defn
    \let\oldenddefn\enddefn
    \renewenvironment{defn}{\begin{framed} \olddefn}{\oldenddefn \end{framed}}
}
\newcommand{\framedtheorems}{
    \let\oldthm\thm
    \let\oldendthm\endthm
    \renewenvironment{thm}{\begin{framed} \oldthm}{\oldendthm \end{framed}}
    \let\oldlem\lem
    \let\oldendlem\endlem
    \renewenvironment{lem}{\begin{framed} \oldlem}{\oldendlem \end{framed}}
    \let\oldprop\prop
    \let\oldendprop\endprop
    \renewenvironment{prop}{\begin{framed} \oldprop}{\oldendprop \end{framed}}
    \let\oldcor\cor
    \let\oldendcor\endcor
    \renewenvironment{cor}{\begin{framed} \oldcor}{\oldendcor \end{framed}}
}
\newcommand{\framedntheorems}{
    \let\oldnthm\nthm
    \let\oldendnthm\endnthm
    \renewenvironment{nthm}{\begin{framed} \oldnthm}{\oldendnthm \end{framed}}
    \let\oldnlem\nlem
    \let\oldendnlem\endnlem
    \renewenvironment{nlem}{\begin{framed} \oldnlem}{\oldendnlem \end{framed}}
    \let\oldnprop\nprop
    \let\oldendnprop\endnprop
    \renewenvironment{nprop}{\begin{framed} \oldnprop}{\oldendnprop
    \end{framed}}
    \let\oldncor\ncor
    \let\oldendncor\endncor
    \renewenvironment{ncor}{\begin{framed} \oldncor}{\oldendncor \end{framed}}
}
%
%-------------------------------------------------------------------------------
%             THE HEADER                                                       |
%-------------------------------------------------------------------------------
%
%   We use the 'fancyhdr' package to create our header. Here I place the <Course
%   Title> on the left, <Homework Title> in the center, and <Your Name> on the
%   right. Since your name will not change between documents, I recommend
%   specifying your name here like I did.
%
\newcommand{\cn}{<Course Title>}
\newcommand{\as}{<Assignment Title>}
\newcommand{\nm}{Tony Mottaz}       % <== REPLACE WITH YOUR NAME
\newcommand{\coursetitle}[1]{\renewcommand{\cn}{#1}}
\newcommand{\hwtitle}[1]{\renewcommand{\as}{#1}}
\newcommand{\myname}[1]{\renewcommand{\nm}{#1}}
\lhead{\scshape \cn}
\chead{\scshape \as}
\rhead{\scshape \nm}
\pagestyle{fancy}
%
%-------------------------------------------------------------------------------
%             UNCATEGORIZED FEATURES                                           |
%-------------------------------------------------------------------------------
%
%----- PROOF ENVIRONMENT -----%
%
%   Here I redefine the proof environment so that if fewer than 4 lines will
%   show at the bottom of a page, then the entire environment will be pushed
%   onto the next page.
\expandafter\let\expandafter\oldproof\csname\string\proof\endcsname
\let\oldendproof\endproof
\renewenvironment{proof}[1][\proofname]{%
    \Needspace*{4\baselineskip} \oldproof[#1]}
    {\oldendproof}
%
%----- DARK THEME -----%
%
%   Here I provide the command '\darktheme' to turn your output PDF completely
%   dark. This lowers eye strain when editing documents for long time periods.
\newcommand{\darktheme}{
    \pagecolor[rgb]{0.1,0.1,0.1}
    \color[rgb]{0.95,0.95,0.95}
}
%
%----- CIRCLED -----%
%
%   If you want something circled, this will do it nicely.
\newcommand*\circled[1]{\tikz[baseline=(char.base)]{%
    \node[shape=ellipse,draw,inner sep=1pt] (char) {#1};}}
%
%----- MARGIN ALIGN -----%
%
%   If you want item labels aligned at the margins, use
%   \begin{enumerate}[align=margin,labelsep=0pt]
\SetLabelAlign{margin}{\llap{#1~~}}
%
%----- MARGIN PARAGRAPHS -----%
\newcommand{\mpar}[1]{\marginpar{\footnotesize#1}}
%
% The following is for use with 'mylatexformat'
%\csname endofdump\endcsname
