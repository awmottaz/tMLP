%%%%%%%%%%%%%%%%%%%%%%%%%%%%%%%%%%%%%%%%%%%%%%%%%%%%%%%%%%%%%%%%%%%%%%%%%%%%%%%%%%%
%% THE MOTTAZ PREAMBLE %%%%%%%%%%%%%%%%%%%%%%%%%%%%%%%%%%%%%%%%%%%%%%%%%%%%%%%%%%%%
%%%%%%%%%%%%%%%%%%%%%%%%%%%%%%%%%%%%%%%%%%%%%%%%%%%%%%%%%%%%%%%%%%%%%%%%%%%%%%%%%%%

%% Created by: Tony Mottaz
%% Description: This is my customized preamble with which I write all of my homework assignments.
%% It is ever-expanding to fit new and nitpicky uses.

%% Version: 1.1
%% Publish Date: 10/27/2015
%% Publish Location: Tony's Copy.com cloud storage account

%% Refer to the README for the change log and other information.


%%%%%%%%%%%%%%%%%%%%%%%%%%%%%%%%%%%%%%%%%%%%%%%%%%%%%%%%%%%%%%%%%%%%%%%%%%%%%%%%%%%
%% SET THE LAYOUT OF THE PAGE %%%%%%%%%%%%%%%%%%%%%%%%%%%%%%%%%%%%%%%%%%%%%%%%%%%%%
%%%%%%%%%%%%%%%%%%%%%%%%%%%%%%%%%%%%%%%%%%%%%%%%%%%%%%%%%%%%%%%%%%%%%%%%%%%%%%%%%%%
% NOTE: This layout assumes that the user is using the letterpaper size. Specify this by using
%		\documentclass[letterpaper]{•} 
%		at the beginning of your .tex document.

\usepackage{layout}
\hoffset 0.25in
\voffset 0pt
\oddsidemargin 0pt
\topmargin 0pt
\headheight 1.3em
\headsep 25pt
\textheight 600pt
\textwidth 6in
\marginparwidth 85pt


%%%%%%%%%%%%%%%%%%%%%%%%%%%%%%%%%%%%%%%%%%%%%%%%%%%%%%%%%%%%%%%%%%%%%%%%%%%%%%%%%%%
%% PACKAGES %%%%%%%%%%%%%%%%%%%%%%%%%%%%%%%%%%%%%%%%%%%%%%%%%%%%%%%%%%%%%%%%%%%%%%%
%%%%%%%%%%%%%%%%%%%%%%%%%%%%%%%%%%%%%%%%%%%%%%%%%%%%%%%%%%%%%%%%%%%%%%%%%%%%%%%%%%%
\usepackage{mathtools,amsthm,amssymb}	% For basic math typesetting needs
\usepackage{dsfont,mathrsfs}			% For double stroke fonts (\mathds{•}) and math script fonts (\mathscr{•})
\usepackage[textwidth=0.9\marginparwidth]{todonotes} % For todo notes in the margin, plus other features (refer to the documentation)
\usepackage{fancyhdr}					% Creates the fancy header
\usepackage{cancel}						% Math package to show cancellation
\usepackage{parskip}					% New paragraphs skip a line rather than indent
\usepackage{enumerate}					% For custom enumerations (refer to documentation)
\usepackage[colorlinks=true,linkcolor=blue,linkbordercolor=white]{hyperref}	% Creates hyperlinks
\usepackage{xcolor}						% Extended coloring options (refer to documentation)
\usepackage{relsize}					% Can change size of math (refer to documentation)
\usepackage{tikz,pgfplots}				% The Tikz package is utilized for creating mathematical images, and PGFPlots provides tools for creating graphs
\pgfplotsset{compat=newest}				% This gives PGFPlots backwards compatibility
\usepackage{framed}						% Framed environment puts a box around its contents (refer to documentation)



%%%%%%%%%%%%%%%%%%%%%%%%%%%%%%%%%%%%%%%%%%%%%%%%%%%%%%%%%%%%%%%%%%%%%%%%%%%%%%%%%%%
%% USER DEFINED FOOTNOTE SYMBOLS %%%%%%%%%%%%%%%%%%%%%%%%%%%%%%%%%%%%%%%%%%%%%%%%%%
%%%%%%%%%%%%%%%%%%%%%%%%%%%%%%%%%%%%%%%%%%%%%%%%%%%%%%%%%%%%%%%%%%%%%%%%%%%%%%%%%%%
% WARNING: If the user needs more footnote symbols on a page than are listed here, there will be an error. To add another symbol, add '\or\symbol' to the string below.

\makeatletter
\newcommand*{\wackyfn}[1]{%
  \expandafter\@wackyfn\csname c@#1\endcsname%
}

\newcommand*{\@wackyfn}[1]{
	%~Add your symbol below this line~%
  $\ifcase#1 \or\dag\or\ddag%
    \else\@ctrerr\fi$
}



%%%%%%%%%%%%%%%%%%%%%%%%%%%%%%%%%%%%%%%%%%%%%%%%%%%%%%%%%%%%%%%%%%%%%%%%%%%%%%%%%%%
%% USER DEFINED COMMANDS & MATH OPERATORS %%%%%%%%%%%%%%%%%%%%%%%%%%%%%%%%%%%%%%%%% %%%%%%%%%%%%%%%%%%%%%%%%%%%%%%%%%%%%%%%%%%%%%%%%%%%%%%%%%%%%%%%%%%%%%%%%%%%%%%%%%%%

% Shortcuts for double stroke characters
\newcommand{\N}{\mathds{N}}
\newcommand{\Z}{\mathds{Z}}
\newcommand{\Q}{\mathds{Q}}
\newcommand{\R}{\mathds{R}}
\newcommand{\C}{\mathds{C}}
\newcommand{\F}{\mathds{F}}

% Various mathematical delimeters and reassignment of operators
\newcommand{\ang}[1]{\left\langle #1 \right\rangle}	% Places the argument inside auto-scaled angular brackets
\newcommand{\abs}[1]{\left\vert #1 \right\vert}		% Places the argument inside auto-scaled vertical bars
\newcommand{\norm}[1]{\left\| #1 \right\|}			% Places the argument inside auto-scaled double vertical bars
\newcommand{\iso}{\cong}							% The 'is isomorphic to' symbol
\newcommand{\nsg}{\unlhd}							% The 'normal subgroup' symbol
\newcommand{\rnsg}{\unrhd}							% Reversed 'normal subgroup' symbol
\newcommand{\nnsg}{\ntrianglelefteq}				% Negated 'normal subgroup' symbol
\newcommand{\del}{\nabla} 							% The 'del', or 'gradient' operator
\newcommand{\dsum}{\displaystyle\sum\limits}		% Like \dfrac, but for inline sums
\renewcommand{\limsup}{\overline{\lim}\,}			% The 'limit superior' symbol
\renewcommand{\liminf}{\underline{\lim}\,}			% The 'limit inferior' symbol
\def\upint{\mathchoice								% Upper integral
    {\mkern13mu\overline{\vphantom{\intop}\mkern7mu}\mkern-20mu}
    {\mkern7mu\overline{\vphantom{\intop}\mkern7mu}\mkern-14mu}
    {\mkern7mu\overline{\vphantom{\intop}\mkern7mu}\mkern-14mu}
    {\mkern7mu\overline{\vphantom{\intop}\mkern7mu}\mkern-14mu}
  \int}
\def\lowint{\mkern3mu\underline{\vphantom{\intop}\mkern7mu}\mkern-10mu\int}		% Lower integral

% User defined fraction-like objects.
% The genfrac command takes 6 arguments.
% The first two arguments are optional left and right delimiters, respectively
% The third argument is an optional specification of the line thickness
% The fourth argument optionally overrides the mathstyle sizing:
% 0--displaystyle, 1--textstyle, 2--scriptstyle, 3--scriptscriptstyle
% The last two arguments are the numerator and denominator, respectively.
% For example, a binomial coefficient would be written in this environment:
% \newcommand{\binom}[2]{\genfrac{(}{)}{0pt}{}{#1}{#2}}

% User defined math operators
\DeclareMathOperator{\ord}{ord}		% Order
\DeclareMathOperator{\sgn}{sgn}		% Sign function
\DeclareMathOperator{\lcm}{lcm}		% Least common multiple
\DeclareMathOperator{\aut}{Aut}		% Group of automorphisms
\DeclareMathOperator{\inn}{Inn}		% Group of inner automorphisms
\DeclareMathOperator{\sym}{Sym}		% Symmetric group
\DeclareMathOperator{\id}{id}		% Identity operator
\DeclareMathOperator{\img}{Im}		% Image of a function
\DeclareMathOperator{\stab}{Stab}	% Stabilizer
\DeclareMathOperator{\orb}{Orb}		% Orbit
\DeclareMathOperator{\cl}{C\ell}	% Conjugacy class
\DeclareMathOperator{\core}{core}	% Core
\DeclareMathOperator{\syl}{Syl}		% Sylow group

% Use the footnote symbols defined above
% Comment out the following two lines to use default footnote symbols
\renewcommand\thefootnote{\wackyfn{footnote}}
\makeatother

% I want math inside of the \textbf{•} command to also be bold
\DeclareTextFontCommand{\textbf}{\boldmath\bfseries}


%%%%%%%%%%%%%%%%%%%%%%%%%%%%%%%%%%%%%%%%%%%%%%%%%%%%%%%%%%%%%%%%%%%%%%%%%%%%%%%%%%%
%% THEOREM STYLES %%%%%%%%%%%%%%%%%%%%%%%%%%%%%%%%%%%%%%%%%%%%%%%%%%%%%%%%%%%%%%%%%
%%%%%%%%%%%%%%%%%%%%%%%%%%%%%%%%%%%%%%%%%%%%%%%%%%%%%%%%%%%%%%%%%%%%%%%%%%%%%%%%%%%

% REFER TO the amsthm documentation for further explanation.

% First define the styles.
\newtheoremstyle{first_style} 		% name
    {\topsep}                    	% Space above
    {\topsep}                    	% Space below
    {\slshape}                		% Body font
    {}                           	% Indent amount
    {\bfseries}                  	% Theorem head font
    {.}              				% Punctuation after theorem head
    {.5em}                       	% Space after theorem head
    {}  						  	% Theorem head spec (can be left empty, meaning *normal*)
    
\newtheoremstyle{second_style} 		% name
    {5pt}                    		% Space above
    {\topsep}                    	% Space below
    {\normalfont}                	% Body font
    {}                           	% Indent amount
    {\bfseries}                  	% Theorem head font
    {:}              				% Punctuation after theorem head
    {.5em}                       	% Space after theorem head
    {}  						  	% Theorem head spec (can be left empty, meaning *normal*)
    
\newtheoremstyle{third_style} 		% name
    {\topsep}                    	% Space above
    {\topsep}                    	% Space below
    {\normalfont}                	% Body font
    {}                           	% Indent amount
    {\slshape}                  	% Theorem head font
    {$\,\rightarrow$}              	% Punctuation after theorem head
    {.5em}                       	% Space after theorem head
    {}  						 	% Theorem head spec (can be left empty, meaning *normal*)
    
% Now define the theorem environments based on these styles.
\theoremstyle{first_style}
\newtheorem*{thm}{Theorem}
\newtheorem*{lem}{Lemma}
\newtheorem*{prop}{Proposition}
\newtheorem*{cor}{Corollary}

\theoremstyle{second_style}
\newtheorem*{defn}{Definition}
\newtheorem*{exmp}{Example}
\newtheorem*{sol}{Solution}

\theoremstyle{third_style}
\newtheorem*{note}{Note}
\newtheorem*{claim}{Claim}


%%%%%%%%%%%%%%%%%%%%%%%%%%%%%%%%%%%%%%%%%%%%%%%%%%%%%%%%%%%%%%%%%%%%%%%%%%%%%%%%%%%
%% DEFINE HEADER %%%%%%%%%%%%%%%%%%%%%%%%%%%%%%%%%%%%%%%%%%%%%%%%%%%%%%%%%%%%%%%%%%
%%%%%%%%%%%%%%%%%%%%%%%%%%%%%%%%%%%%%%%%%%%%%%%%%%%%%%%%%%%%%%%%%%%%%%%%%%%%%%%%%%%
\lhead{<Course Title>}
\chead{<Assignment Title>}
\rhead{Tony Mottaz}		% Replace with your name
\pagestyle{fancy}
