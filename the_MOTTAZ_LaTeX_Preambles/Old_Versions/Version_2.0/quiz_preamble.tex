%%%%%%%%%%%%%%%%%%%%%%%%%%%%%%%%%%%%%%%%%%%%%%%%%%%%%%%%%%%%%%%%%%%%%%%%%%%%%%%%
%%%%%%%%%%%%%%%%%% the MOTTAZ STANDARD LaTeX PREAMBLE (QUIZ) %%%%%%%%%%%%%%%%%%%
%%%%%%%%%%%%%%%%%%%%%%%%%%%%%%%%%%%% v. 2.0 %%%%%%%%%%%%%%%%%%%%%%%%%%%%%%%%%%%%
%
%    Author: Anthony Mottaz
%    Version: 1.0  (2016/02/10)
%    Description: The MOTTAZ STANDARD LaTeX PREAMBLE is a collection of
%                 preambles that have been optimally designed for tasks that are
%                 of primary concern to students and professors of mathematics.
%                    This particular preamble, the 'QUIZ' edition, has been
%                 optimized for creating quizzes. For a complete list and
%                 descriptions of these features, refer to the Usage Guide PDF
%                 document. For release notes and version change logs, see the
%                 README.
%                    Lots of time and thought has gone into creating the most
%                 comprehensive and useful preambles possible, but the work is
%                 never done. If you experience issues with this preamble or
%                 have any suggestions for improvement, please contact me at
%
%                                 anthonywmottaz@gmail.com
%
%                 Happy TeXing!
%                 - A.M.
%
%    HOW TO USE: At the beginning of your document, after '\documentclass[•]{•}'
%                add '%%%%%%%%%%%%%%%%%%%%%%%%%%%%%%%%%%%%%%%%%%%%%%%%%%%%%%%%%%%%%%%%%%%%%%%%%%%%%%
%% THE MOTTAZ QUIZ LATEX PREAMBLE %%%%%%%%%%%%%%%%%%%%%%%%%%%%%%%%%%%%%%%
%%%%%%%%%%%%%%%%%%%%%%%%%%%%%%%%%%%%%%%%%%%%%%%%%%%%%%%%%%%%%%%%%%%%%%%%%%%%%%

%% Created by: Tony Mottaz
%% Description: This is the first spinoff of the Mottaz Universal
%% LaTeX Preamble. This preamble is designed and optimized for creating
%% quizzes.

%% Version: 1.0
%% Publish Date: January 26, 2015
%% Publish Location: Tony's Copy.com cloud storage account


%%%%%%%%%%%%%%%%%%%%%%%%%%%%%%%%%%%%%%%%%%%%%%%%%%%%%%%%%%%%%%%%%%%%%%%%%%%%%%
%% SET THE LAYOUT OF THE PAGE %%%%%%%%%%%%%%%%%%%%%%%%%%%%%%%%%%%%%%%%%%%%%%%%
%%%%%%%%%%%%%%%%%%%%%%%%%%%%%%%%%%%%%%%%%%%%%%%%%%%%%%%%%%%%%%%%%%%%%%%%%%%%%%

% The 'layout' package gives the command '\layout' which, when used in the
%	beginning of your document, gives a summary of the various layout
%	measurements.
\usepackage{layout}

% We now customize the layout measurements. These are designed to work for any
%	paper size, e.g. a4paper, letterpaper, legalpaper, etc.
%	All dimensions are calculated based only on the size of the page and the
%	size of the font. Refer to the Usage Guide for the motivation behind this
%	definition.

\hoffset -0.03\paperwidth
\voffset -0.04\paperheight
\oddsidemargin 0pt
\topmargin 0pt
\headheight 1.3em
\headsep 1.3em
\textheight 0.78\paperheight
\textwidth 0.82\paperwidth
\marginparsep 1.5em
\marginparwidth 0.17\paperwidth
\footskip 2.5\headheight



%%%%%%%%%%%%%%%%%%%%%%%%%%%%%%%%%%%%%%%%%%%%%%%%%%%%%%%%%%%%%%%%%%%%%%%%%%%%%%
%% CHOOSE AN ALTERNATIVE FONT %%%%%%%%%%%%%%%%%%%%%%%%%%%%%%%%%%%%%%%%%%%%%%%%
%%%%%%%%%%%%%%%%%%%%%%%%%%%%%%%%%%%%%%%%%%%%%%%%%%%%%%%%%%%%%%%%%%%%%%%%%%%%%%

\usepackage{lmodern}%			By default, we use the Latin Modern font
\usepackage[T1]{fontenc}

\newcommand{\concfont}{%		The classic font created by Donald Knuth
\usepackage{concmath}
\usepackage[T1]{fontenc}
}

\newcommand{\stixfont}{%		Similar to Times, good for textbook-style text
\usepackage[T1]{fontenc}
\usepackage{stix}
}

%%%%%%%%%%%%%%%%%%%%%%%%%%%%%%%%%%%%%%%%%%%%%%%%%%%%%%%%%%%%%%%%%%%%%%%%%%%%%%
%% PACKAGES %%%%%%%%%%%%%%%%%%%%%%%%%%%%%%%%%%%%%%%%%%%%%%%%%%%%%%%%%%%%%%%%%%
%%%%%%%%%%%%%%%%%%%%%%%%%%%%%%%%%%%%%%%%%%%%%%%%%%%%%%%%%%%%%%%%%%%%%%%%%%%%%%

% For basic math typesetting needs
\usepackage{mathtools,amsthm,amssymb}

% For double stroke fonts (\mathds{•}) and math script fonts (\mathscr{•})
\usepackage{dsfont,mathrsfs}

% Math package to show cancellation
\usepackage{cancel}

% For custom enumerations
% The 'shortlabels' option allows for the use of 'enumerate'-like definitions
%	of the labels.
\usepackage[shortlabels]{enumitem}

% For additional customization of tabular and array environments.
\usepackage{array}
% The 'array' package allows us to cure the problem of capitol letters
%	touching the '\hline's in a tabular environment.
\setlength{\extrarowheight}{1pt}

% This package allows the user to insert dashed lines in a tabular/array
%	environments.
% For vertical lines, use the ':' character in lieu of '|' as a column
%	separator.
% For horizontal lines, use the commands '\hdashline' and '\cdashline' in
%	place of '\hline' and 'cline'.
\usepackage{arydshln}

% Can change size of math
\usepackage{relsize}

% Extended coloring options
\usepackage[dvipsnames]{xcolor}

% The Tikz package is utilized for creating mathematical images, and PGFPlots 
% provides tools for creating graphs
\usepackage{tikz,pgfplots}
% This gives PGFPlots backwards compatibility
\pgfplotsset{compat = 1.10}
% This allows the user to use the PGFPlots feature 'fillbetween'
\usepgfplotslibrary{fillbetween}

% To correct proof environment spacing (see below)
\usepackage{needspace}

% For inserting todo notes and placeholders for missing figures.
\usepackage[textwidth=0.9\marginparwidth]{todonotes}

% Creates the fancy header
\usepackage{fancyhdr}

% New paragraphs skip a line rather than indent
\usepackage{parskip}

% Creates hyperlinks to section headings and labelled items.
\usepackage[
	colorlinks=true,
	linkcolor=blue,
	linkbordercolor=white,
	urlcolor=blue,
	unicode
	]{hyperref}

% Additional hyperlink possibilities
\usepackage{cleveref}

% Framed environment puts a box around its contents
\usepackage{framed}

% For some miscellaneous symbols
\usepackage{wasysym}

% For adding dummy text
\usepackage{lipsum}

% Utilized in footnote symbols definition (see below)
\usepackage{alphalph}

% For inserting PDF documents
\usepackage{pdfpages}


% Used in definition of '\signchart'
\usepackage{xstring}

% Extra control over figures. Use the 'H' placement character to mean
%	"here, no matter what"
\usepackage{float}

% For customizing tabular environments.
\usepackage{tabularx}

% These two packages allow for writing upright greek letters in text mode and
%	math mode, respectively, using commands such as '\textalpha',
%	'\textAlpha', '\upalpha', and '\Upalpha'.
\usepackage{textgreek,upgreek}

% For fancy customized verbatim environments
\usepackage{fancyvrb}

% For creating key-value pairs, used in the definition of '\signchart'
\usepackage{xkeyval}



%%%%%%%%%%%%%%%%%%%%%%%%%%%%%%%%%%%%%%%%%%%%%%%%%%%%%%%%%%%%%%%%%%%%%%%%%%%%%%
%% USER DEFINED COMMANDS & MATH OPERATORS %%%%%%%%%%%%%%%%%%%%%%%%%%%%%%%%%%%%
%%%%%%%%%%%%%%%%%%%%%%%%%%%%%%%%%%%%%%%%%%%%%%%%%%%%%%%%%%%%%%%%%%%%%%%%%%%%%%

% Shortcuts for double stroke characters
% Note: One of the packages defines a command '\C'. I don't know which one
%	does it, nor do I know what the command does. So I am just ignoring it and
%	redefining it for my needs.
\newcommand{\N}{\mathds{N}}		% Naturals
\newcommand{\Z}{\mathds{Z}}		% Integers
\newcommand{\Q}{\mathds{Q}}		% Rationals
\newcommand{\R}{\mathds{R}}		% Reals
\renewcommand{\C}{\mathds{C}}	% Complex numbers
\renewcommand{\H}{\mathds{H}}	% Quaternions (Hamiltonions)
\newcommand{\F}{\mathds{F}}		% Generic field

%%%%% Various mathematical delimeters and reassignment of operators %%%%%%%%%%

% Places the argument inside auto-scaled parentheses
\newcommand{\paren}[1]{\left( #1 \right)}

% Places the argument inside auto-scaled angular brackets
\newcommand{\ang}[1]{\left\langle #1 \right\rangle}

% Places the argument inside auto-scaled braces
\newcommand{\brc}[1]{\left\{ #1 \right\}}

% Places the argument inside auto-scaled vertical bars
\newcommand{\abs}[1]{\left\vert #1 \right\vert}

% Places the argument inside auto-scaled double vertical bars
\newcommand{\norm}[1]{\left\| #1 \right\|}


\newcommand{\ceil}[1]{\left\lceil #1 \right\rceil}		% Ceiling function
\newcommand{\floor}[1]{\left\lfloor #1 \right\rfloor}	% Floor function


\newcommand{\iso}{\cong}	% The 'is isomorphic to' symbol
\newcommand{\nsg}{\unlhd}	% The 'normal subgroup' symbol
\newcommand{\rnsg}{\unrhd}	% Reversed 'normal subgroup' symbol
\newcommand{\nnsg}{\ntrianglelefteq}	% Negated 'normal subgroup' symbol
\newcommand{\del}{\nabla} 	% The 'del', or 'gradient' operator
\newcommand{\dsum}{\displaystyle\sum\limits}	% Like \dfrac, but for inline
%	sums
\renewcommand{\limsup}{\overline{\lim}\,}	% The 'limit superior' symbol
\renewcommand{\liminf}{\underline{\lim}\,}	% The 'limit inferior' symbol

% Upper integral
\def\upint{\mathchoice
    {\mkern13mu\overline{\vphantom{\intop}\mkern7mu}\mkern-20mu}
    {\mkern7mu\overline{\vphantom{\intop}\mkern7mu}\mkern-14mu}
    {\mkern7mu\overline{\vphantom{\intop}\mkern7mu}\mkern-14mu}
    {\mkern7mu\overline{\vphantom{\intop}\mkern7mu}\mkern-14mu}
  \int}
% Lower integral
\def\lowint{\mkern3mu\underline{\vphantom{\intop}\mkern7mu}\mkern-10mu\int}

% I redefine the bar command in order to
%	1. make it slightly wider, and
%	2. auto-scale over larger inputs.
\renewcommand{\bar}[1]{
\mkern 2mu\overline{\mkern-2mu#1\mkern-2mu}\mkern 2mu}

%%%%% User defined fraction-like objects: %%%%%%%%%%%%%%%%%%%%%%%%%%%%%%%%%%%%
% The genfrac command takes 6 arguments.
% The first two arguments are optional left and right delimiters, respectively
% The third argument is an optional specification of the line thickness
% The fourth argument optionally overrides the mathstyle sizing.
% Use the following keys:
% 0--displaystyle, 1--textstyle, 2--scriptstyle, 3--scriptscriptstyle
% The fifth and sixth arguments are the numerator and denominator,
%	respectively.
% For example, a binomial coefficient would be created with this definition:
% \newcommand{\binom}[2]{\genfrac{(}{)}{0pt}{}{#1}{#2}}

%%%%% Math Operators %%%%%%%%%%%%%%%%%%%%%%%%%%%%%%%%%%%%%%%%%%%%%%%%%%%%%%%%%
% Define some math operators
\DeclareMathOperator{\ord}{ord}		% Order
\DeclareMathOperator{\sgn}{sgn}		% Sign function
\DeclareMathOperator{\lcm}{lcm}		% Least common multiple
\DeclareMathOperator{\aut}{Aut}		% Group of automorphisms
\DeclareMathOperator{\inn}{Inn}		% Group of inner automorphisms
\DeclareMathOperator{\sym}{Sym}		% Symmetric group
\DeclareMathOperator{\id}{id}		% Identity operator
\DeclareMathOperator{\img}{Im}		% Image of a function
\DeclareMathOperator{\stab}{Stab}	% Stabilizer
\DeclareMathOperator{\orb}{Orb}		% Orbit
\DeclareMathOperator{\cl}{C\ell}	% Conjugacy class
\DeclareMathOperator{\core}{core}	% Core
\DeclareMathOperator{\syl}{Syl}		% Sylow group
\DeclareMathOperator{\cha}{char}	% Characteristic
\DeclareMathOperator{\tr}{tr}		% Trace of a matrix
\DeclareMathOperator{\fun}{Fun}		% Set of functions
\DeclareMathOperator{\cis}{cis}		% Cos + i Sin
\DeclareMathOperator{\Arg}{Arg}		% Principal argument

% Redefine \Re and \Im to give 'Re' and 'Im', respectively, in place of the
%	standard fraktur font versions
\renewcommand{\Re}{\operatorname{Re}}
\renewcommand{\Im}{\operatorname{Im}}


%%%%% Other Math-related Commands %%%%%%%%%%%%%%%%%%%%%%%%%%%%%%%%%%%%%%%%%%%%
% I want math inside of the \textbf{•} command to also be bold
\DeclareTextFontCommand{\textbf}{\boldmath\bfseries}

% This command autmatically generates a sign chart.
% Syntax: \signchart{<list of values>}{<list of signs>}
% Example: \signchart{1,2,3}{+,-,+,-}
% Options: 'width' and 'height' change the width of the chart
%	and the height of the symbols, respectively.

\makeatletter
\define@key{signchart}{height}[0.3]{\def\signHeight{#1}}
\define@key{signchart}{width}[5]{\def\signChartWidth{#1}}
\define@key{signchart}{arrows}[<->]{\def\theArrow{#1}}
\makeatother

\newcommand{\signchart}[3][]{
\setkeys{signchart}{height, width, arrows, #1}
\begin{center}
\begin{tikzpicture}

% Define the distance signs are from number line
\pgfmathsetmacro{\snht}{\signHeight}

% Define the total width of the sign chart
\pgfmathsetmacro{\wid}{\signChartWidth}

\def\nums{#2}	% Read in the values of the sign chart
\def\signs{#3}	% Read in the string array of signs
\def\numsarray{{\nums}}

% Determine number of values minus one
\StrCount{\nums}{,}[\len]

\draw[\theArrow,thick] (0,0) -- (\wid,0);

\foreach \i in {0,...,\len} {
	
	% Extract current number
	\pgfmathparse{\numsarray[\i]}
	\let\thisnum\pgfmathresult
	
	% Extract current sign
	\pgfmathtruncatemacro{\j}{\i + 1}
	\StrBehind[\j]{\signs}{,}[\stepOne]
	\StrLen{\stepOne}[\aLength]
	\pgfmathtruncatemacro{\cutAmount}{\aLength + 1}
	\StrGobbleRight{\signs}{\cutAmount}[\almost]
	\StrBehind[\j]{,\almost}{,}[\s]
	
	% Calculate horizontal positions of number and sign
	\pgfmathsetmacro{\numpos}{(\wid/(\len+2))*(\i+1)}
	\pgfmathsetmacro{\signcoord}{(\wid/(\len+2))*(\i+0.5)}
	
	% Draw a tick mark with the number above it.
	\draw (\numpos,-0.15) -- (\numpos,0.15) node[anchor=south] {\thisnum};
	% Add the sign to the left of the tick mark.
	\node at (\signcoord,\snht) {$\s$};
	}

% The sign to the right of the last tick mark is left out of the loop, so we
% now calculate its horizontal position and add it to the sign chart.
\pgfmathsetmacro{\signcoord}{(\wid/(\len+2))*(\len+1.5)}
\pgfmathtruncatemacro{\j}{\len + 1}
\StrBehind[\j]{\signs}{,}[\s]
\node at (\signcoord,\snht) {$\s$};

\end{tikzpicture}
\end{center}
}



%%%%%%%%%%%%%%%%%%%%%%%%%%%%%%%%%%%%%%%%%%%%%%%%%%%%%%%%%%%%%%%%%%%%%%%%%%%%%%
%% THEOREM STYLES %%%%%%%%%%%%%%%%%%%%%%%%%%%%%%%%%%%%%%%%%%%%%%%%%%%%%%%%%%%%
%%%%%%%%%%%%%%%%%%%%%%%%%%%%%%%%%%%%%%%%%%%%%%%%%%%%%%%%%%%%%%%%%%%%%%%%%%%%%%

% REFER TO the amsthm documentation for further explanation.

% First define the styles.
\newtheoremstyle{second_style} 		% name
    {\baselineskip}            		% Space above
    {0.7\baselineskip}             	% Space below
    {\normalfont}                	% Body font
    {}                           	% Indent amount
    {\bfseries}                  	% Theorem head font
    {:}              				% Punctuation after theorem head
    {.5em}                       	% Space after theorem head
    {}  						  	% Theorem head spec   
    
% Now define the theorem environments based on these styles.
\theoremstyle{second_style}
	\newtheorem*{defn}{Definition}
	\newtheorem*{exmp}{Example}
	\newtheorem*{sol}{Solution}
	\newtheorem*{case}{Case}

% Note: I have chosen short names for these in order to save keystrokes.



%%%%%%%%%%%%%%%%%%%%%%%%%%%%%%%%%%%%%%%%%%%%%%%%%%%%%%%%%%%%%%%%%%%%%%%%%%%%%%
%% DEFINE HEADER %%%%%%%%%%%%%%%%%%%%%%%%%%%%%%%%%%%%%%%%%%%%%%%%%%%%%%%%%%%%%
%%%%%%%%%%%%%%%%%%%%%%%%%%%%%%%%%%%%%%%%%%%%%%%%%%%%%%%%%%%%%%%%%%%%%%%%%%%%%%

% Use the commands '\coursetitle{•}' and '\quiztitle{•}' to set the course
% title and quiz titles, respectively.

\def\cn{<Course Title>}
\def\qt{<Quiz Title>}

\newcommand{\coursetitle}[1]{\def\cn{#1}}
\newcommand{\quiztitle}[1]{\def\as{#1}}

% Set the pagestyle to suppress page numbers.
\pagestyle{empty}

% Commands to populate the top of the page.
\newcommand{\makeheader}{
	\textbf{\cn} \hfill \textbf{Name:} \rule{3in}{1pt}
	
	\textbf{\as}}

%%%%%%%%%%%%%%%%%%%%%%%%%%%%%%%%%%%%%%%%%%%%%%%%%%%%%%%%%%%%%%%%%%%%%%%%%%%%%%
%% OTHER FEATURES %%%%%%%%%%%%%%%%%%%%%%%%%%%%%%%%%%%%%%%%%%%%%%%%%%%%%%%%%%%%
%%%%%%%%%%%%%%%%%%%%%%%%%%%%%%%%%%%%%%%%%%%%%%%%%%%%%%%%%%%%%%%%%%%%%%%%%%%%%%

% For referencing multiple items to single footnote,
% 	use '\footnote{\label{foo}<text in the footnote>}',
% 	and later use: '\cref{foo}'
\crefformat{footnote}{#2\footnotemark[#1]#3}



% The following command inverts the colors of the PDF output.
% This is nice while editing to lower eye strain.
% Note: This command requires the xcolor package.
\newcommand{\darktheme}{
\pagecolor[rgb]{0.1,0.1,0.1}
\color[rgb]{0.95,0.95,0.95}
}



% This command allows inline numbers to be placed inside a circle.
% Note: This command requires the tikz package.
\newcommand*\circled[1]{\tikz[baseline=(char.base)]{
    \node[shape=circle,draw,inner sep=1pt] (char) {#1};}}


 
% Use the option 'align=margin' to place item labels in the left margin.
% Add the option 'labelsep=0pt' so that the content in the items is
%	properly aligned.
% Requires the 'enumitem' package.
% Example: \begin{enumerate}[align=margin,labelsep=0pt]
\SetLabelAlign{margin}{\llap{#1~~}}


% This command allows the user to specify the number of points a particular
% question is worth.
\newcommand{\points}[1]{\texttt{(#1 points)}}' where
%                'path/to/quiz_preamble.tex'
%                is the location of the 'quiz_preamble.tex' file on your system.
%
%-------------------------------------------------------------------------------
%             PAGE LAYOUT                                                      |
%-------------------------------------------------------------------------------
%
\usepackage{layout}
%
%   Use the command '\layout{}' after '\begin{document}' to print a page showing
%   the current layout settings. This is useful if you wish to adjust these
%   measurements.
%
%   Next we customize these dimensions. Notice that every dimension calculation
%   is based on either the size of the page or the size of the font. This allows
%   for compatibility across different user preferences and setups. Refer to the
%   Usage Guide for the motivation behind these choices.
\hoffset -0.03\paperwidth
\voffset -0.04\paperheight
\oddsidemargin 0pt
\topmargin 0pt
\headheight 1.3em
\headsep 1.3em
\textheight 0.78\paperheight
\textwidth 0.82\paperwidth
\marginparsep 1.5em
\marginparwidth 0.17\paperwidth
\footskip 2.5\headheight
%
%-------------------------------------------------------------------------------
%             LOAD ALL OF THE PACKAGES                                         |
%-------------------------------------------------------------------------------
%
%   This is where we load all of the packages that may be useful in creating
%   your document. Refer to the Usage Guide for thorough information on the
%   application of each package.
%
\usepackage{lmodern}
\usepackage[T1]{fontenc}
\usepackage[utf8]{inputenc}
\usepackage{mathtools}
\usepackage{amsthm}
\usepackage{amssymb}
\usepackage{dsfont}
\usepackage{mathrsfs}
\usepackage{cancel}
\usepackage[shortlabels]{enumitem}
\usepackage{array}
	\setlength{\extrarowheight}{1pt}
\usepackage{arydshln}
\usepackage{relsize}
\usepackage[dvipsnames]{xcolor}
\usepackage{tikz}
	\usetikzlibrary{shapes}
\usepackage{pgfplots}
	\pgfplotsset{compat = 1.10}
	\usepgfplotslibrary{fillbetween}
\usepackage{needspace}
\usepackage[textwidth = 0.9\marginparwidth]{todonotes}
\usepackage{fancyhdr}
\usepackage{parskip}
\usepackage{imakeidx}
\usepackage[
	colorlinks=true,
	linkcolor=blue,
	linkbordercolor=white,
	urlcolor=blue,
	unicode
	]{hyperref}
\usepackage{cleveref}
\usepackage{framed}
\usepackage{wasysym}
\usepackage{lipsum}
\usepackage{alphalph}
\usepackage{pdfpages}
\usepackage{float}
\usepackage{tabularx}
\usepackage{textgreek}
\usepackage{upgreek}
\usepackage{fancyvrb}
\usepackage{signchart}
\usepackage{microtype}
\usepackage{multicol}
%
%-------------------------------------------------------------------------------
%             MATH-SPECIFIC COMMANDS, OPERATORS, AND DEFINITIONS               |
%-------------------------------------------------------------------------------
%
%----- DOUBLE STROKE CHARACTERS -----%
\newcommand{\N}{\mathds{N}}		% Naturals
\newcommand{\Z}{\mathds{Z}}		% Integers
\newcommand{\Q}{\mathds{Q}}		% Rationals
\newcommand{\R}{\mathds{R}}		% Reals
\renewcommand{\C}{\mathds{C}}	% Complex numbers
\renewcommand{\H}{\mathds{H}}	% Quaternions (Hamiltonions)
\newcommand{\F}{\mathds{F}}		% Generic field
%
%----- AUTOMATICALLY SCALED DELIMITERS -----%
\newcommand{\paren}[1]{\left( #1 \right)}%                              ( )
\newcommand{\ang}[1]{\left\langle #1 \right\rangle}%                    < >
\newcommand{\brc}[1]{\left\{ #1 \right\}}%                              { }
\newcommand{\brkt}[1]{\left[ #1 \right]}%                               [ ]
\newcommand{\abs}[1]{\left\vert #1 \right\vert}%                        | |
\newcommand{\norm}[1]{\left\| #1 \right\|}%                            || ||
%
%----- FUNCTIONS -----%
\newcommand{\ceil}[1]{\left\lceil #1 \right\rceil}		% Ceiling function
\newcommand{\floor}[1]{\left\lfloor #1 \right\rfloor}	% Floor function
%
%----- OPERATORS -----%
\newcommand{\iso}{\cong}	                   % The 'is isomorphic to' symbol
\newcommand{\nsg}{\unlhd}                  % The 'normal subgroup' symbol
\newcommand{\rnsg}{\unrhd}                 % Reversed 'normal subgroup' symbol
\newcommand{\nnsg}{\ntrianglelefteq}	       % Negated 'normal subgroup' 
%symbol
\newcommand{\del}{\nabla}                  % The 'del', or 'gradient' operator
\renewcommand{\bar}[1]{                    % The complex conjugate
	\mkern 1mu\overline{\mkern-1mu#1\mkern-1mu}}
\DeclareMathOperator{\ord}{ord}            % Order
\DeclareMathOperator{\sgn}{sgn}            % Sign function
\DeclareMathOperator{\lcm}{lcm}            % Least common multiple
\DeclareMathOperator{\aut}{Aut}            % Group of automorphisms
\DeclareMathOperator{\inn}{Inn}            % Group of inner automorphisms
\DeclareMathOperator{\sym}{Sym}            % Symmetric group
\DeclareMathOperator{\id}{id}              % Identity operator
\DeclareMathOperator{\img}{Im}             % Image of a function
\DeclareMathOperator{\stab}{Stab}          % Stabilizer
\DeclareMathOperator{\orb}{Orb}            % Orbit
\DeclareMathOperator{\cl}{C\ell}	           % Conjugacy class
\DeclareMathOperator{\core}{core}          % Core
\DeclareMathOperator{\syl}{Syl}            % Sylow group
\DeclareMathOperator{\cha}{char}           % Characteristic
\DeclareMathOperator{\tr}{tr}              % Trace of a matrix
\DeclareMathOperator{\fun}{Fun}            % Set of functions
\DeclareMathOperator{\cis}{cis}            % Cos + i Sin
\DeclareMathOperator{\Arg}{Arg}            % Principal argument
\renewcommand{\limsup}{\varlimsup}         % Limit superior
\renewcommand{\liminf}{\varliminf}         % Limit inferior
\renewcommand{\Re}{\operatorname{Re}}      % Real part
\renewcommand{\Im}{\operatorname{Im}}      % Imaginary part
% Upper integral
\def\upint{\mathchoice
    {\mkern13mu\overline{\vphantom{\intop}\mkern7mu}\mkern-20mu}
    {\mkern7mu\overline{\vphantom{\intop}\mkern7mu}\mkern-14mu}
    {\mkern7mu\overline{\vphantom{\intop}\mkern7mu}\mkern-14mu}
    {\mkern7mu\overline{\vphantom{\intop}\mkern7mu}\mkern-14mu}
	\int}
% Lower integral
\def\lowint{\mkern3mu\underline{\vphantom{\intop}\mkern7mu}\mkern-10mu\int}
\newcommand{\dsum}{\displaystyle\sum\limits}% Like \dfrac, but for inline sums
%
%----- UNCATEGORIZED -----%
%   I want math inside of \textbf{•} to also be bolded font
\DeclareTextFontCommand{\textbf}{\boldmath\bfseries}
%
%   You can create your own fraction-like commands using '\genfrac'. This takes
%   6 inputs:
%      1. left delimiter
%      2. right delimiter
%      3. fraction bar line thickness
%      4. mathstyle size override:
%         0--displaystyle, 1--textstyle, 2--scriptstyle, 3--scriptscriptstyle
%      5. numerator content
%      6. denominator content
%
%-------------------------------------------------------------------------------
%             THE HEADER                                                       |
%-------------------------------------------------------------------------------
%
%   Use the commands '\coursetitle{•}' and '\quiztitle{•}' to set the course
%   title and quiz titles, respectively. These are placed on the left side at
%   the top of the page, and a blank 'Name' space on the right.
%
\def\cn{<Course Title>}
\def\qt{<Quiz Title>}
\newcommand{\coursetitle}[1]{\def\cn{#1}}
\newcommand{\quiztitle}[1]{\def\as{#1}}
% Set the pagestyle to suppress page numbers.
\pagestyle{empty}
% Commands to populate the top of the page.
\newcommand{\makeheader}{
	\textbf{\cn} \hfill \textbf{Name:} \rule{3in}{1pt} \par
	\textbf{\as}}
%
%-------------------------------------------------------------------------------
%             UNCATEGORIZED FEATURES                                           |
%-------------------------------------------------------------------------------
%
%----- DARK THEME -----%
%
%   Here I provide the command '\darktheme' to turn your output PDF completely
%   dark. This lowers eye strain when editing documents for long time periods.
\newcommand{\darktheme}{
	\pagecolor[rgb]{0.1,0.1,0.1}
	\color[rgb]{0.95,0.95,0.95}
}
%
%----- CIRCLED -----%
%
%   If you want something circled, this will do it nicely.
\newcommand*\circled[1]{\tikz[baseline=(char.base)]{%
	\node[shape=ellipse,draw,inner sep=1pt] (char) {#1};}}
%
%----- MARGIN ALIGN -----%
%
%   If you want item labels aligned at the margins, use
%   \begin{enumerate}[align=margin,labelsep=0pt]
\SetLabelAlign{margin}{\llap{#1~~}}
%
%----- MARGIN PARAGRAPHS -----%
\newcommand{\mpar}[1]{\marginpar{\footnotesize#1}}
%
%----- POINTS -----%
%
%   Use the '\points{#}' command to tell the quiz-taker how many points a
%   certain question is worth.
\newcommand{\points}[1]{\texttt{(#1 points)}}
