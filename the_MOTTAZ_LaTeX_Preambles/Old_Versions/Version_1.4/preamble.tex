%%%%%%%%%%%%%%%%%%%%%%%%%%%%%%%%%%%%%%%%%%%%%%%%%%%%%%%%%%%%%%%%%%%%%%%%%%%%%%%%%%%
%% THE MOTTAZ UNIVERSAL LATEX PREAMBLE %%%%%%%%%%%%%%%%%%%%%%%%%%%%%%%%%%%%%%%%%%%%
%%%%%%%%%%%%%%%%%%%%%%%%%%%%%%%%%%%%%%%%%%%%%%%%%%%%%%%%%%%%%%%%%%%%%%%%%%%%%%%%%%%

%% Created by: Tony Mottaz
%% Description: This is my customized preamble with which I write all of my homework 
%% assignments.
%% It is ever-expanding to fit new and nitpicky uses.
%% Please refer to the README for change log, goals for the future, and other info.

%% Version: 1.4
%% Publish Date: December 5, 2015
%% Publish Location: Tony's Copy.com cloud storage account


%%%%%%%%%%%%%%%%%%%%%%%%%%%%%%%%%%%%%%%%%%%%%%%%%%%%%%%%%%%%%%%%%%%%%%%%%%%%%%%%%%%
%% SET THE LAYOUT OF THE PAGE %%%%%%%%%%%%%%%%%%%%%%%%%%%%%%%%%%%%%%%%%%%%%%%%%%%%%
%%%%%%%%%%%%%%%%%%%%%%%%%%%%%%%%%%%%%%%%%%%%%%%%%%%%%%%%%%%%%%%%%%%%%%%%%%%%%%%%%%%

% The 'layout' package gives the command '\layout' which, when used in the
%	beginning of your document, gives a summary of the various layout
%	measurements.
\usepackage{layout}

% We now customize the layout measurements. These are designed to work for any
%	paper size, e.g. a4paper, letterpaper, legalpaper, etc.
%	All dimensions are calculated based only on the size of the page and the
%	size of the font. Refer to the README for the motivation behind this
%	definition.

\hoffset 0.06\paperwidth
\voffset -0.025\paperheight
\oddsidemargin 0pt
\topmargin 0pt
\headheight 1.3em
\headsep 1.3em
\textheight 0.78\paperheight
\textwidth 0.59\paperwidth
\marginparsep 1.5em
\marginparwidth 0.17\paperwidth
\footskip 2.5\headheight



%%%%%%%%%%%%%%%%%%%%%%%%%%%%%%%%%%%%%%%%%%%%%%%%%%%%%%%%%%%%%%%%%%%%%%%%%%%%%%%%%%%
%% CHOOSE AN ALTERNATIVE FONT %%%%%%%%%%%%%%%%%%%%%%%%%%%%%%%%%%%%%%%%%%%%%%%%%%%%%
%%%%%%%%%%%%%%%%%%%%%%%%%%%%%%%%%%%%%%%%%%%%%%%%%%%%%%%%%%%%%%%%%%%%%%%%%%%%%%%%%%%

\newcommand{\concfont}{%		The classic font created by Donald Knuth
\usepackage{concmath}
\usepackage[T1]{fontenc}
}

\newcommand{\stixfont}{%		Similar to Times, good for textbook-style text
\usepackage[T1]{fontenc}
\usepackage{stix}
}


%%%%%%%%%%%%%%%%%%%%%%%%%%%%%%%%%%%%%%%%%%%%%%%%%%%%%%%%%%%%%%%%%%%%%%%%%%%%%%%%%%%
%% PACKAGES %%%%%%%%%%%%%%%%%%%%%%%%%%%%%%%%%%%%%%%%%%%%%%%%%%%%%%%%%%%%%%%%%%%%%%%
%%%%%%%%%%%%%%%%%%%%%%%%%%%%%%%%%%%%%%%%%%%%%%%%%%%%%%%%%%%%%%%%%%%%%%%%%%%%%%%%%%%

% For basic math typesetting needs
\usepackage{mathtools,amsthm,amssymb}

% For double stroke fonts (\mathds{•}) and math script fonts (\mathscr{•})
\usepackage{dsfont,mathrsfs}

% Math package to show cancellation
\usepackage{cancel}

% For custom enumerations (refer to documentation)
\usepackage{enumerate}

% Can change size of math (refer to documentation)
\usepackage{relsize}

% Extended coloring options (refer to documentation)
\usepackage[dvipsnames]{xcolor}

% The Tikz package is utilized for creating mathematical images, and PGFPlots 
% provides tools for creating graphs
\usepackage{tikz,pgfplots}
% This gives PGFPlots backwards compatibility
\pgfplotsset{compat=1.10}
% This allows the user to use the PGFPlots feature 'fillbetween'
\usepgfplotslibrary{fillbetween}

% To correct proof environment spacing (see below)
\usepackage{needspace}

% For todo notes in the margin, plus other features (refer to the documentation)
\usepackage[textwidth=0.9\marginparwidth]{todonotes}

% Creates the fancy header
\usepackage{fancyhdr}

% New paragraphs skip a line rather than indent
\usepackage{parskip}

% Creates hyperlinks
\usepackage[colorlinks=true,linkcolor=blue,linkbordercolor=white]{hyperref}

% Additional hyperlink possibilities
\usepackage{cleveref}

% Framed environment puts a box around its contents (refer to documentation)
\usepackage{framed}

% For some miscellaneous symbols
\usepackage{wasysym}

% For adding dummy text
\usepackage{lipsum}

% Utilized in footnote symbols definition (see below)
\usepackage{alphalph}

% For inserting PDF documents
\usepackage{pdfpages}


% Used in definition of '\signchart'
\usepackage{xstring}

% Extra control over figures. Use the 'H' placement character to mean
%	"here, no matter what"
\usepackage{float}



%%%%%%%%%%%%%%%%%%%%%%%%%%%%%%%%%%%%%%%%%%%%%%%%%%%%%%%%%%%%%%%%%%%%%%%%%%%%%%%%%%%
%% USER DEFINED FOOTNOTE SYMBOLS %%%%%%%%%%%%%%%%%%%%%%%%%%%%%%%%%%%%%%%%%%%%%%%%%%
%%%%%%%%%%%%%%%%%%%%%%%%%%%%%%%%%%%%%%%%%%%%%%%%%%%%%%%%%%%%%%%%%%%%%%%%%%%%%%%%%%%
% NOTE: To add another symbol, add '\or\symbol' to the string below,
% as demonstrated.

\makeatletter
\newcommand*{\myfnsymbolsingle}[1]{%
  \ensuremath{%
    \ifcase#1
    \or
      \dag 
    \or
      \ddag
    \or
      \kreuz
    \or
      \star
    \else
      \@ctrerr  
    \fi
  }%   
}   
\makeatother

\newcommand*{\myfnsymbol}[1]{%
  \myfnsymbolsingle{\value{#1}}%
}

% We can remove the upper boundary counting error 
% by multiplying the symbols.

\newalphalph{\myfnsymbolmult}[mult]{\myfnsymbolsingle}{}

\renewcommand*{\thefootnote}{%
  \myfnsymbolmult{\value{footnote}}%
}


%%%%%%%%%%%%%%%%%%%%%%%%%%%%%%%%%%%%%%%%%%%%%%%%%%%%%%%%%%%%%%%%%%%%%%%%%%%%%%%%%%%
%% USER DEFINED COMMANDS & MATH OPERATORS %%%%%%%%%%%%%%%%%%%%%%%%%%%%%%%%%%%%%%%%%
%%%%%%%%%%%%%%%%%%%%%%%%%%%%%%%%%%%%%%%%%%%%%%%%%%%%%%%%%%%%%%%%%%%%%%%%%%%%%%%%%%%

% Shortcuts for double stroke characters
\newcommand{\N}{\mathds{N}}
\newcommand{\Z}{\mathds{Z}}
\newcommand{\Q}{\mathds{Q}}
\newcommand{\R}{\mathds{R}}
\newcommand{\C}{\mathds{C}}
\newcommand{\F}{\mathds{F}}

%% Various mathematical delimeters and reassignment of operators %%

% Places the argument inside auto-scaled angular brackets
\newcommand{\ang}[1]{\left\langle #1 \right\rangle}

% Places the argument inside auto-scaled vertical bars
\newcommand{\abs}[1]{\left\vert #1 \right\vert}

% Places the argument inside auto-scaled double vertical bars
\newcommand{\norm}[1]{\left\| #1 \right\|}

\newcommand{\iso}{\cong}	% The 'is isomorphic to' symbol
\newcommand{\nsg}{\unlhd}	% The 'normal subgroup' symbol
\newcommand{\rnsg}{\unrhd}	% Reversed 'normal subgroup' symbol
\newcommand{\nnsg}{\ntrianglelefteq}	% Negated 'normal subgroup' symbol
\newcommand{\del}{\nabla} 	% The 'del', or 'gradient' operator
\newcommand{\dsum}{\displaystyle\sum\limits}	% Like \dfrac, but for inline sums
\renewcommand{\limsup}{\overline{\lim}\,}	% The 'limit superior' symbol
\renewcommand{\liminf}{\underline{\lim}\,}	% The 'limit inferior' symbol

% Upper integral
\def\upint{\mathchoice
    {\mkern13mu\overline{\vphantom{\intop}\mkern7mu}\mkern-20mu}
    {\mkern7mu\overline{\vphantom{\intop}\mkern7mu}\mkern-14mu}
    {\mkern7mu\overline{\vphantom{\intop}\mkern7mu}\mkern-14mu}
    {\mkern7mu\overline{\vphantom{\intop}\mkern7mu}\mkern-14mu}
  \int}
% Lower integral
\def\lowint{\mkern3mu\underline{\vphantom{\intop}\mkern7mu}\mkern-10mu\int}
\newcommand{\eref}[1]{(\ref{#1})}

% User defined fraction-like objects:
% The genfrac command takes 6 arguments.
% The first two arguments are optional left and right delimiters, respectively
% The third argument is an optional specification of the line thickness
% The fourth argument optionally overrides the mathstyle sizing.
% Use the following keys:
% 0--displaystyle, 1--textstyle, 2--scriptstyle, 3--scriptscriptstyle
% The fifth and sixth arguments are the numerator and denominator, respectively.
% For example, a binomial coefficient would be created with this definition:
% \newcommand{\binom}[2]{\genfrac{(}{)}{0pt}{}{#1}{#2}}

% Define some math operators
\DeclareMathOperator{\ord}{ord}		% Order
\DeclareMathOperator{\sgn}{sgn}		% Sign function
\DeclareMathOperator{\lcm}{lcm}		% Least common multiple
\DeclareMathOperator{\aut}{Aut}		% Group of automorphisms
\DeclareMathOperator{\inn}{Inn}		% Group of inner automorphisms
\DeclareMathOperator{\sym}{Sym}		% Symmetric group
\DeclareMathOperator{\id}{id}		% Identity operator
\DeclareMathOperator{\img}{Im}		% Image of a function
\DeclareMathOperator{\stab}{Stab}	% Stabilizer
\DeclareMathOperator{\orb}{Orb}		% Orbit
\DeclareMathOperator{\cl}{C\ell}	% Conjugacy class
\DeclareMathOperator{\core}{core}	% Core
\DeclareMathOperator{\syl}{Syl}		% Sylow group
\DeclareMathOperator{\cha}{char}	% Characteristic
\DeclareMathOperator{\tr}{tr}		% Trace of a matrix

% I want math inside of the \textbf{•} command to also be bold
\DeclareTextFontCommand{\textbf}{\boldmath\bfseries}

% For referencing multiple items to single footnote,
% use '\footnote{\label{foo}<text in the footnote>}',
% and later use: '\cref{foo}'
\crefformat{footnote}{#2\footnotemark[#1]#3}

% The following command inverts the colors of the PDF output.
% This is nice while editing to lower eye strain.
% Note: This command requires the xcolor package.
\newcommand{\darktheme}{
\pagecolor[rgb]{0.1,0.1,0.1}
\color[rgb]{0.95,0.95,0.95}
}

% This command allows inline numbers to be placed inside a circle.
% Note: This command requires the tikz package.
\newcommand*\circled[1]{\tikz[baseline=(char.base)]{
    \node[shape=circle,draw,inner sep=1pt] (char) {#1};}}
    
% This command autmatically generates a signchart.
% Syntax: \signchart{<list of values>}{<array of symbols inside of double quotes>}
% Example: \signchart{1,2,3}{{"$+$","$-$","$+$","$-$"}}
% Note the double braces in the second argument
\newcommand{\signchart}[2]{
\begin{center}
\begin{tikzpicture}

\def\yval{0.3}	% In the future, this will be a keyval option
\pgfmathparse{\yval}
\let\y\pgfmathresult
\def\pwidth{5}	% In the future, this will be a keyval option
\pgfmathparse{\pwidth}
\let\wid\pgfmathresult

\def\signs{#2}	% Read in the string array of signs
\def\mylist{#1}	% Read in the values of the sign chart
\def\other{{\mylist}}
\StrCount{\mylist}{,}[\len]


\draw[<->,thick] (0,0) -- (\wid,0);

\foreach \i in {0,...,\len} {
	% First we need to calculate the location of each mark on the number line,
	% as well as the location for each sign.
	\pgfmathparse{\other[\i]}
	\let\x\pgfmathresult
	\pgfmathparse{(\wid/(\len+2))*(\i+1)}
	\let\j\pgfmathresult
	\pgfmathparse{(\wid/(\len+2))*(\i+0.5)}
	\let\k\pgfmathresult
	\pgfmathparse{\signs[\i]}
	\let\s\pgfmathresult
	
	% Now we draw the mark with the number above it.
	\draw (\j,-0.15) -- (\j,0.15) node[anchor=south] {\x};
	% Now add the sign to the left of the mark.
	\node at (\k,\y) {\s};
	}

% The sign to the right of the last number is left out of the loop, so we
% now calculate its position and add it.
\pgfmathparse{(\wid/(\len+2))*(\len+1.5)}
\let\k\pgfmathresult
\pgfmathparse{\signs[\len+1]}
\let\s\pgfmathresult
\node at (\k,\y) {\s};

\end{tikzpicture}
\end{center}
}


%%%%%%%%%%%%%%%%%%%%%%%%%%%%%%%%%%%%%%%%%%%%%%%%%%%%%%%%%%%%%%%%%%%%%%%%%%%%%%%%%%%
%% THEOREM STYLES %%%%%%%%%%%%%%%%%%%%%%%%%%%%%%%%%%%%%%%%%%%%%%%%%%%%%%%%%%%%%%%%%
%%%%%%%%%%%%%%%%%%%%%%%%%%%%%%%%%%%%%%%%%%%%%%%%%%%%%%%%%%%%%%%%%%%%%%%%%%%%%%%%%%%

% REFER TO the amsthm documentation for further explanation.

% First define the styles.
\newtheoremstyle{first_style} 		% name
    {\topsep}                    	% Space above
    {\topsep}                    	% Space below
    {\slshape}                		% Body font
    {}                           	% Indent amount
    {\bfseries}                  	% Theorem head font
    {.}              				% Punctuation after theorem head
    {.5em}                       	% Space after theorem head
    {}  						  	% Theorem head spec (can be left empty, meaning *normal*)
    
\newtheoremstyle{second_style} 		% name
    {0.5ex}                    		% Space above
    {\topsep}                    	% Space below
    {\normalfont}                	% Body font
    {}                           	% Indent amount
    {\bfseries}                  	% Theorem head font
    {:}              				% Punctuation after theorem head
    {.5em}                       	% Space after theorem head
    {}  						  	% Theorem head spec (can be left empty, meaning *normal*)
    
\newtheoremstyle{third_style} 		% name
    {1ex}	                    	% Space above
    {-0.5\baselineskip}   	            % Space below
    {\normalfont}                	% Body font
    {}                           	% Indent amount
    {\slshape \bfseries}            % Theorem head font
    {$\,\rightarrow \,$}           	% Punctuation after theorem head
    {0pt}                       	% Space after theorem head
    {}  						 	% Theorem head spec (can be left empty, meaning *normal*)
    
    
% Now define the theorem environments based on these styles.
\theoremstyle{first_style}
	% Numbered versions:
	\newtheorem{nthm}{Theorem}[section]
	\newtheorem{nlem}[nthm]{Lemma}
	\newtheorem{nprop}[nthm]{Proposition}
	\newtheorem{ncor}[nthm]{Corollary}

	% Non-numbered versions:
	\newtheorem*{thm}{Theorem}
	\newtheorem*{lem}{Lemma}
	\newtheorem*{prop}{Proposition}
	\newtheorem*{cor}{Corollary}

\theoremstyle{second_style}
	\newtheorem*{defn}{Definition}
	\newtheorem*{exmp}{Example}
	\newtheorem*{sol}{Solution}

\theoremstyle{third_style}
	\newtheorem*{note}{Note}
	\newtheorem*{claim}{Claim}

% Note: I have chosen short names for these to save keystrokes.


%%%%%%%%%%%%%%%%%%%%%%%%%%%%%%%%%%%%%%%%%%%%%%%%%%%%%%%%%%%%%%%%%%%%%%%%%%%%%%%%%%%
%% PROOF ENVIRONMENT SPACING %%%%%%%%%%%%%%%%%%%%%%%%%%%%%%%%%%%%%%%%%%%%%%%%%%%%%%
%%%%%%%%%%%%%%%%%%%%%%%%%%%%%%%%%%%%%%%%%%%%%%%%%%%%%%%%%%%%%%%%%%%%%%%%%%%%%%%%%%%

% I prefer a proof to start on a new page if 3 lines or fewer of the proof will
% only be seen on a single page. The 'needspace' package does this for me
% quite nicely.

% First, copy '\proof' and '\endproof' to avoid infinite loop errors:
\expandafter\let\expandafter\oldproof\csname\string\proof\endcsname
\let\oldendproof\endproof

% Now we can redefine the proof environment:
\renewenvironment{proof}[1][\proofname]{%
	\needspace{4\baselineskip} \oldproof[#1]}
	{\oldendproof}


%%%%%%%%%%%%%%%%%%%%%%%%%%%%%%%%%%%%%%%%%%%%%%%%%%%%%%%%%%%%%%%%%%%%%%%%%%%%%%%%%%%
%% DEFINE HEADER %%%%%%%%%%%%%%%%%%%%%%%%%%%%%%%%%%%%%%%%%%%%%%%%%%%%%%%%%%%%%%%%%%
%%%%%%%%%%%%%%%%%%%%%%%%%%%%%%%%%%%%%%%%%%%%%%%%%%%%%%%%%%%%%%%%%%%%%%%%%%%%%%%%%%%

% Use the commands '\coursetitle{•}', '\hwtitle{•}', and '\myname{•}'
% to populate the header.
% It is recommended to set your name in the preamble as shown here
% so that you do not need to set it every time.

\def\cn{<Course Title>}
\def\as{<Assignment Title>}
\def\nm{Tony Mottaz}	% Replace with your name

\newcommand{\coursetitle}[1]{\def\cn{#1}}
\newcommand{\hwtitle}[1]{\def\as{#1}}
\newcommand{\myname}[1]{\def\nm{#1}}

\lhead{\cn}
\chead{\as}
\rhead{\nm}

% Lastly, we set the pagestyle to utilize the 'fancyhdr' package.
\pagestyle{fancy}