\documentclass[letterpaper]{ltxdoc}
%%%%%%%%%%%%%%%%%%%%%%%%%%%%%%%%%%%%%%%%%%%%%%%%%%%%%%%%%%%%%%%%%%%%%%%%%%%%%%%%
%%%%%%%%%%%%%%%%%%%%%%%% the MOTTAZ LaTeX PREAMBLE (NOTES) %%%%%%%%%%%%%%%%%%%%%
%%%%%%%%%%%%%%%%%%%%%%%%%%%%%%%%%%%% v. 1.00 %%%%%%%%%%%%%%%%%%%%%%%%%%%%%%%%%%%
%
%    Author: Anthony Mottaz
%    Version: 1.00  (2016/02/10)
%    Description: The MOTTAZ LaTeX PREAMBLES are a collection of preambles that
%                 have been optimally designed for tasks that are of primary
%                 concern to students and professors of mathematics.
%                    This particular preamble, the 'NOTES' edition, has lots of
%                 preloaded packages and features that are useful for typing
%                 course notes for a mathematics class. For a complete list and
%                 descriptions of these features, refer to the Usage Guide PDF
%                 document. For release notes and version change logs, see
%                 'notes_changelog.log'.
%                    Lots of time and thought has gone into creating the most
%                 comprehensive and useful preamble possible, but the work is
%                 never done. If you experience issues with this preamble or
%                 have any suggestions for improvement, please contact me at
%
%                                 anthonywmottaz@gmail.com
%
%                 Happy TeXing!
%                 - A.M.
%
%    HOW TO USE: At the beginning of your document, after '\documentclass[•]{•}'
%                add '\input{path/to/notes_preamble.tex}' where
%                'path/to/notes_preamble.tex'
%                is the location of the 'notes_preamble.tex' file on your
%                system.
%
%    LICENSE:    This project is released under the LaTeX Project Public
%                License, v1.3c or later. See
%
%                          http://www.latex-project.org/lppl.txt
%
%-------------------------------------------------------------------------------
%             PAGE LAYOUT                                                      |
%-------------------------------------------------------------------------------
%
\usepackage{layout}
%
%   The following section defines the page layout for the document. Notice that
%   every defined length is based on either the size of the page or the size of 
%   the font. This is so that any choice in paper size and font results in a
%   similar layout. To understand the motivation behind these dimension choices,
%   refer to the Usage Guide.
%
\usepackage{geometry}
\geometry{
    headheight = 1.3em,
    headsep = 1.3em,
    marginparsep = 1.5em,
    footnotesep = 3em,
    hdivide={0.15\paperwidth,0.62\paperwidth,*},
    vdivide={0.1\paperheight,0.78\paperheight,*},
    marginparwidth=0.18\paperwidth
}
%
%-------------------------------------------------------------------------------
%             LOAD ALL OF THE PACKAGES                                         |
%-------------------------------------------------------------------------------
%
%   This is where we load all of the packages that may be useful in creating
%   your document. Refer to the Usage Guide for thorough information on the
%   application of each package.
%
%   WARNING: Do not change the order in which these packages are loaded. Doing
%            so may result in errors.
%
\usepackage{lmodern}
\usepackage[T1]{fontenc}
\usepackage[utf8]{inputenc}
\usepackage{mathtools}
\usepackage{amsthm}
\usepackage{amssymb}
\usepackage{dsfont}
\usepackage{mathrsfs}
\usepackage{cancel}
\usepackage[shortlabels]{enumitem}
\usepackage{array}
	\setlength{\extrarowheight}{1pt}
	\usepackage{longtable}
\usepackage{arydshln}
\usepackage{relsize}
\usepackage[dvipsnames]{xcolor}
\usepackage{tikz}
	\usetikzlibrary{shapes}
    \usetikzlibrary{arrows}
\usepackage{pgfplots}
	\pgfplotsset{compat = 1.10}% You may use a later version, if you wish.
	\usepgfplotslibrary{fillbetween}
\usepackage{needspace}
\usepackage[textwidth = 0.9\marginparwidth]{todonotes}
\usepackage{fancyhdr}
\usepackage[parfill]{parskip}
\usepackage{imakeidx}
\usepackage[
	colorlinks=true,
	linkcolor=blue,
	linkbordercolor=white,
	urlcolor=blue,
	unicode
	]{hyperref}
\usepackage{cleveref}
\usepackage{framed}
\usepackage{wasysym}
\usepackage{lipsum}
\usepackage{alphalph}
\usepackage{pdfpages}
\usepackage{float}
\usepackage{tabularx}
\usepackage{textgreek}
\usepackage{upgreek}
\usepackage{signchart}
\usepackage{microtype}
\usepackage{multicol}
\usepackage{tikz-cd}
\usepackage{fancyvrb}

%
%-------------------------------------------------------------------------------
%             FOOTNOTE SYMBOLS                                                 |
%-------------------------------------------------------------------------------
%
%   Since the intended use of this preamble is for the creating of mathematical
%   documents, we must be careful that the footnote symbols will not be confused
%   with mathematical symbols, e.g. a '2' footnote mark being mistaken as the
%   square of a number. This section redefines the footnote symbols. To add your
%   own symbol to the list, simply append
%     \or
%       \newsymbol
%   to the list below.
%
\makeatletter%
\newcommand*{\myfnsymbolsingle}[1]{%
  \ensuremath{%
    \ifcase#1
    \or
      \dag
    \or
      \ddag
    \or
      \kreuz
    \or
      \star
    \else
      \@ctrerr  
    \fi
  }%   
}   
\makeatother%
\newcommand*{\myfnsymbol}[1]{%
  \myfnsymbolsingle{\value{#1}}%
}
\newalphalph{\myfnsymbolmult}[mult]{\myfnsymbolsingle}{}%
\renewcommand*{\thefootnote}{%
  \myfnsymbolmult{\value{footnote}}%
}
%
% For referencing multiple locations to single footnote, use
%   '\footnote{\label{foo}<text in the footnote>}',
% 	and later use: '\cref{foo}'
%
\crefformat{footnote}{#2\footnotemark[#1]#3}%
%
%-------------------------------------------------------------------------------
%             MATH-SPECIFIC COMMANDS, OPERATORS, AND DEFINITIONS               |
%-------------------------------------------------------------------------------
%
%----- DOUBLE STROKE CHARACTERS -----%
\newcommand{\N}{\mathds{N}}		% Naturals
\newcommand{\Z}{\mathds{Z}}		% Integers
\newcommand{\Q}{\mathds{Q}}		% Rationals
\newcommand{\R}{\mathds{R}}		% Reals
\renewcommand{\C}{\mathds{C}}	% Complex numbers
\renewcommand{\H}{\mathds{H}}	% Quaternions (Hamiltonions)
\newcommand{\F}{\mathds{F}}		% Generic field
%
%----- AUTOMATICALLY SCALED DELIMITERS -----%
\DeclarePairedDelimiter{\paren}{(}{)}%                                  ( )
\DeclarePairedDelimiter{\ang}{\langle}{\rangle}%                        < >
\DeclarePairedDelimiter{\brc}{\{}{\}}%                                  { }
\DeclarePairedDelimiter{\brkt}{[}{]}%                                   [ ]
\DeclarePairedDelimiter{\abs}{\lvert}{\rvert}%                          | |
\DeclarePairedDelimiter{\norm}{\lVert}{\rVert}%                        || ||
\renewcommand{\mid}{\mkern4mu\middle\vert\mkern4mu}
%
% Swap the definition of starred and non-starred versions.
% Non-starred delimeters will auto-scale, and starred versions will not.
\makeatletter
% \paren
\let\oldparen\paren
\def\paren{\@ifstar{\oldparen}{\oldparen*}}
% \ang
\let\oldang\ang
\def\ang{\@ifstar{\oldang}{\oldang*}}
% \brc
\let\oldbrc\brc
\def\brc{\@ifstar{\oldbrc}{\oldbrc*}}
% \brkt
\let\oldbrkt\brkt
\def\brkt{\@ifstar{\oldbrkt}{\oldbrkt*}}
% \abs
\let\oldabs\abs
\def\abs{\@ifstar{\oldabs}{\oldabs*}}
% \norm
\let\oldnorm\norm
\def\norm{\@ifstar{\oldnorm}{\oldnorm*}}
\makeatother
%
%----- FUNCTIONS -----%
\newcommand{\ceil}[1]{\left\lceil #1 \right\rceil}		% Ceiling function
\newcommand{\floor}[1]{\left\lfloor #1 \right\rfloor}	% Floor function
\renewcommand{\bar}[1]{                                 % The complex conjugate
	\mkern 1mu\overline{\mkern-1mu#1\mkern-1mu}}
%
%----- OPERATORS -----%
\newcommand{\iso}{\cong}	               % The 'is isomorphic to' symbol
\newcommand{\nsg}{\unlhd}                  % The 'normal subgroup' symbol
\newcommand{\rnsg}{\unrhd}                 % Reversed 'normal subgroup' symbol
\newcommand{\nnsg}{\ntrianglelefteq}	   % Negated 'normal subgroup' symbol
\newcommand{\del}{\nabla}                  % The 'del', or 'gradient' operator
\DeclareMathOperator{\ord}{ord}            % Order
\DeclareMathOperator{\sgn}{sgn}            % Sign function
\DeclareMathOperator{\lcm}{lcm}            % Least common multiple
\DeclareMathOperator{\aut}{Aut}            % Group of automorphisms
\DeclareMathOperator{\inn}{Inn}            % Group of inner automorphisms
\DeclareMathOperator{\sym}{Sym}            % Symmetric group
\DeclareMathOperator{\id}{id}              % Identity operator
\DeclareMathOperator{\img}{Im}             % Image of a function
\DeclareMathOperator{\stab}{Stab}          % Stabilizer
\DeclareMathOperator{\orb}{Orb}            % Orbit
\DeclareMathOperator{\cl}{C\ell}	       % Conjugacy class
\DeclareMathOperator{\core}{core}          % Core
\DeclareMathOperator{\syl}{Syl}            % Sylow group
\DeclareMathOperator{\cha}{char}           % Characteristic
\DeclareMathOperator{\tr}{tr}              % Trace of a matrix
\DeclareMathOperator{\fun}{Fun}            % Set of functions
\DeclareMathOperator{\cis}{cis}            % Cos + i Sin
\DeclareMathOperator{\Arg}{Arg}            % Principal argument
\DeclareMathOperator{\Frac}{Frac}          % Field of fractions
\DeclareMathOperator{\ann}{Ann}            % Annihilator
\DeclareMathOperator{\tor}{Tor}            % Torsion set
\DeclareMathOperator{\End}{End}            % Set of endomorphisms
\renewcommand{\limsup}{\varlimsup}         % Limit superior
\renewcommand{\liminf}{\varliminf}         % Limit inferior
\renewcommand{\Re}{\operatorname{Re}}      % Real part
\renewcommand{\Im}{\operatorname{Im}}      % Imaginary part
% Upper integral
\def\upint{\mathchoice
    {\mkern13mu\overline{\vphantom{\intop}\mkern7mu}\mkern-20mu}
    {\mkern7mu\overline{\vphantom{\intop}\mkern7mu}\mkern-14mu}
    {\mkern7mu\overline{\vphantom{\intop}\mkern7mu}\mkern-14mu}
    {\mkern7mu\overline{\vphantom{\intop}\mkern7mu}\mkern-14mu}
	\int}
% Lower integral
\def\lowint{\mkern3mu\underline{\vphantom{\intop}\mkern7mu}\mkern-10mu\int}
\newcommand{\dsum}{\displaystyle\sum\limits}% Like \dfrac, but for inline sums
% Big 'boxplus' symbol
\newcommand{\bigboxplus}{%
  \mathop{%
    \mathchoice{\dobigboxplus\Large}%
               {\dobigboxplus\large}
               {\dobigboxplus\normalsize}
               {\dobigboxplus\small}
    }\displaylimits
}
%
\newcommand{\dobigboxplus}[1]{%
\vcenter{#1\kern.2ex\hbox{$\boxplus$}\kern.2ex}}
% 'Divides' relation
\newcommand{\divides}{\bigm|}
\newcommand{\ndivides}{%
  \mathrel{\mkern.5mu % small adjustment
    % superimpose \nmid to \big|
    \ooalign{\hidewidth$\big|$\hidewidth\cr$\nmid$\cr}%
  }%
}
%
%----- UNCATEGORIZED -----%
%   I want math inside of \textbf{•} to also be bolded font
\DeclareTextFontCommand{\textbf}{\boldmath\bfseries}
%
%   You can create your own fraction-like commands using '\genfrac'. This takes
%   6 inputs:
%      1. left delimiter
%      2. right delimiter
%      3. fraction bar line thickness
%      4. mathstyle size override:
%         0--displaystyle, 1--textstyle, 2--scriptstyle, 3--scriptscriptstyle
%      5. numerator content
%      6. denominator content
%
%-------------------------------------------------------------------------------
%             THEOREM STYLES                                                   |
%-------------------------------------------------------------------------------
%
%   Here we define some custom styles for theorem-like environments. Refer to
%   the 'amsthm' documentation for an explanation of how this is done.
%
%----- DEFINE THE STYLES -----%
\newtheoremstyle{first_style}
	{\baselineskip}
	{0.7\baselineskip}
	{\slshape}
	{}
	{\bfseries}
	{.}
	{0.5em}
	{}
\newtheoremstyle{second_style}
	{\baselineskip}
	{0.7\baselineskip}
	{\normalfont}
	{}
	{\bfseries}
	{:}
	{0.5em}
	{}
\newtheoremstyle{third_style}
	{1ex}
	{-0.5\baselineskip}
	{\normalfont}
	{}
	{\slshape\bfseries}
	{:}
	{0.25em}
	{}
%----- DEFINE THE ENVIRONMENTS -----%
\theoremstyle{first_style}
	\newtheorem{nthm}{Theorem}[section]
	\newtheorem{nlem}[nthm]{Lemma}
	\newtheorem{nprop}[nthm]{Proposition}
	\newtheorem{ncor}[nthm]{Corollary}
	\newtheorem*{thm}{Theorem}
	\newtheorem*{lem}{Lemma}
	\newtheorem*{prop}{Proposition}
	\newtheorem*{cor}{Corollary}
\theoremstyle{second_style}
	\newtheorem*{defn}{Definition}
	\newtheorem*{exmp}{Example}
	\newtheorem*{sol}{Solution}
	\newtheorem*{case}{Case}
\theoremstyle{third_style}
	\newtheorem*{note}{Note}
	\newtheorem*{claim}{Claim}
%----- OPTIONAL FRAMED VERSIONS -----%
\newcommand{\frameddefinitions}{
	\let\olddefn\defn
	\let\oldenddefn\enddefn
	\renewenvironment{defn}{\begin{framed} \olddefn}{\oldenddefn \end{framed}}
}
\newcommand{\framedtheorems}{
	\let\oldthm\thm
	\let\oldendthm\endthm
	\renewenvironment{thm}{\begin{framed} \oldthm}{\oldendthm \end{framed}}
	\let\oldlem\lem
	\let\oldendlem\endlem
	\renewenvironment{lem}{\begin{framed} \oldlem}{\oldendlem \end{framed}}
	\let\oldprop\prop
	\let\oldendprop\endprop
	\renewenvironment{prop}{\begin{framed} \oldprop}{\oldendprop \end{framed}}
	\let\oldcor\cor
	\let\oldendcor\endcor
	\renewenvironment{cor}{\begin{framed} \oldcor}{\oldendcor \end{framed}}
}
\newcommand{\framedntheorems}{
	\let\oldnthm\nthm
	\let\oldendnthm\endnthm
	\renewenvironment{nthm}{\begin{framed} \oldnthm}{\oldendnthm \end{framed}}
	\let\oldnlem\nlem
	\let\oldendnlem\endnlem
	\renewenvironment{nlem}{\begin{framed} \oldnlem}{\oldendnlem \end{framed}}
	\let\oldnprop\nprop
	\let\oldendnprop\endnprop
	\renewenvironment{nprop}{\begin{framed} \oldnprop}{\oldendnprop
	\end{framed}}
	\let\oldncor\ncor
	\let\oldendncor\endncor
	\renewenvironment{ncor}{\begin{framed} \oldncor}{\oldendncor \end{framed}}
}
%
%-------------------------------------------------------------------------------
%             THE HEADER                                                       |
%-------------------------------------------------------------------------------
%
%   We use the 'fancyhdr' package to create our header. Here I place the <Course
%   Title> on the left, <Homework Title> in the center, and <Your Name> on the
%   right. Since your name will not change between documents, I recommend
%   specifying your name here like I did.
%
\newcommand{\cn}{<Course Title>}
\newcommand{\as}{<Assignment Title>}
\newcommand{\nm}{Tony Mottaz}       % <== REPLACE WITH YOUR NAME
\newcommand{\coursetitle}[1]{\renewcommand{\cn}{#1}}
\newcommand{\hwtitle}[1]{\renewcommand{\as}{#1}}
\newcommand{\myname}[1]{\renewcommand{\nm}{#1}}
\lhead{\scshape \cn}
\chead{\scshape \as}
\rhead{\scshape \nm}
\pagestyle{fancy}
%
%-------------------------------------------------------------------------------
%             UNCATEGORIZED FEATURES                                           |
%-------------------------------------------------------------------------------
%
%----- PROOF ENVIRONMENT -----%
%
%   Here I redefine the proof environment so that if fewer than 4 lines will
%   show at the bottom of a page, then the entire environment will be pushed
%   onto the next page.
\expandafter\let\expandafter\oldproof\csname\string\proof\endcsname
\let\oldendproof\endproof
\renewenvironment{proof}[1][\proofname]{%
	\Needspace*{4\baselineskip} \oldproof[#1]}
	{\oldendproof}
%
%----- DARK THEME -----%
%
%   Here I provide the command '\darktheme' to turn your output PDF completely
%   dark. This lowers eye strain when editing documents for long time periods.
\newcommand{\darktheme}{
	\pagecolor[rgb]{0.1,0.1,0.1}
	\color[rgb]{0.95,0.95,0.95}
}
%
%----- CIRCLED -----%
%
%   If you want something circled, this will do it nicely.
\newcommand*\circled[1]{\tikz[baseline=(char.base)]{%
	\node[shape=ellipse,draw,inner sep=1pt] (char) {#1};}}
%
%----- MARGIN ALIGN -----%
%
%   If you want item labels aligned at the margins, use
%   \begin{enumerate}[align=margin,labelsep=0pt]
\SetLabelAlign{margin}{\llap{#1~~}}
%
%----- MARGIN PARAGRAPHS -----%
\newcommand{\mpar}[1]{\marginpar{\footnotesize#1}}
%%%%%%%%%%%%%%%%
%%%%%%%%%%%%%%%%
\chead{the \textsc{Mottaz} \LaTeX\ \textsc{Preambles}}
\lhead{}
\rhead{}
% Title
\title{the \textsc{Mottaz} \LaTeX\ \textsc{Preambles}}
\author{Anthony Mottaz \thanks{\texttt{anthonywmottaz@gmail.com}}}
\date{Last Update: \today}
% Other packages and setup
\usepackage{hologo}
\hypersetup{%
    pdftitle={the MOTTAZ LaTeX PREAMBLES},
    pdfauthor={Anthony Mottaz},
    pdfsubject={LaTeX for students of mathematics},
    pdfcreator={Anthony Mottaz},
    pdfproducer={Anthony Mottaz},
    pdfkeywords={TeX, LaTeX, Mathematics}
}
\newgeometry{
    hdivide={0.2\paperwidth,0.62\paperwidth,*},
    vdivide={0.1\paperheight,0.78\paperheight,*},
    marginparwidth=1.5in,
    headheight=1.3em
}
\definecolor{shadecolor}{RGB}{242,242,200}
\newcommand{\myrule}[1]{\textcolor{#1}{\rule{0.7cm}{8pt}}}
\newcommand{\DescribePackage}[2]{\Needspace*{2\baselineskip}
\marginpar{\raggedleft\vspace{\baselineskip}\href{#1}{\texttt{#2}}}
\label{pkg:#2}
}
%
% The document
\begin{document}
%\layout{}
%% TITLE PAGE %%
\thispagestyle{plain}
\maketitle
\begin{abstract}
The following document gives a detailed overview of the collection of \LaTeX\ 
documents that comprise the \textsc{Mottaz} preambles. At this time, 
this collection includes the following documents:
\begin{itemize}
\item \texttt{HOMEWORK} edition (|homework_preamble.tex|)
\item \texttt{QUIZ} edition (|quiz_preamble.tex|)
\item \texttt{NOTES} edition (|notes_preamble.tex|)
\end{itemize}
Each document is essentially a standalone preamble. They each define a page 
layout, load packages, and define useful commands that help create documents 
for their intended end use.

Along with the \texttt{.tex} files listed above are corresponding \texttt{.fmt} 
files, compiled using the 
\href{http://mirror.hmc.edu/ctan/macros/latex/contrib/mylatexformat/mylatexformat.pdf}
{\texttt{mylatexformat}} tool. Look ahead in the \hyperref[sec:usepream]{Using 
a Preamble} section.
\end{abstract}
\pdfbookmark[1]{\contentsname}{toc}
\tableofcontents
%% MOTIVATION
\section{Motivation}
The beginning of every \LaTeX\ document has a preamble. If you write lots of 
documents in \LaTeX , chances are you have created your own dedicated 
\texttt{preamble.tex} file which you load into every document. As a graduate 
student in mathematics, I started making my own. As I discovered more and more 
ways to customize my document and make handy macros, my preamble started to 
grow. This preamble has become extremely useful for me, and I have put a lot of 
thought and effort into creating a comprehensive collection of packages and 
creation of macros which serve the purposes of math students and professors 
well.

These kinds of projects are not new. Well-meaning students and professors alike 
have created their own preambles that they share with their friends. This 
practice becomes devastating when the preambles become long and disorganized, 
and then some error creeps in and they have no idea how to fix it. This project 
stands out because I have been dedicated to careful documentation from the 
beginning. Every change and iteration has been noted, old versions saved, and 
each preamble is written with high levels of organization and readability in 
mind.

The \texttt{HOMEWORK} edition was the first that I created. It is designed to 
provide any packages one might need to typeset all mathematics homework for any 
mathematics course, and it defines many macros (i.e. commands) to speed up the 
typesetting process. The \texttt{QUIZ} edition came next, and this provides a 
nice layout and some other macros for simplifying the creation of quizzes in 
mathematics courses. The \texttt{NOTES} edition was a shoot-off of the 
\texttt{HOMEWORK} edition, and it is used to take course notes.

If you have any suggestions for changes/additions, or if you have 
problems/errors of any kind, please send me an email at:
\begin{center}
\texttt{anthonywmottaz@gmail.com}
\end{center}
%% FUTURE DEVELOPMENT GOALS
\section{Future Development Goals}
\begin{enumerate}[\bfseries I.]
\item Repackage this collection of preambles into a classes.
\begin{enumerate}[\bfseries i.]
    \item Key-value options: \texttt{oneside} and \texttt{twoside} options; 
    \texttt{microtype} on/off
\end{enumerate}
\item Other versions I would like to make:
\begin{enumerate}[\bfseries i.]
    \item Tests/exams
    \item Solution guides
\end{enumerate}
\item Add a command like |\layout| to print a list of commands or a help guide 
for this preamble.
\end{enumerate}
%% USING A PREAMBLE
\section{Using a Preamble}
\label{sec:usepream}
The following subsections will explain how to use one of the \textsc{Mottaz} 
preambles in your own \LaTeX\ document.
%% LOADING A PREAMBLE
\subsection{Loading a Preamble}
Save the file (|<edition>_preamble.tex|) to your system, and make note of the 
file path. For example, you might save this file to
    \begin{verbatim}
    C:Users/YourName/Documents/<edition>_preamble.tex      (Windows)
    Users/yourname/Documents/<edition>_preamble.tex        (Macintosh)
    /home/yourname/Documents/<edition>_preamble.tex        (Linux)
    \end{verbatim}
In the preamble of the document you are creating, simply use the command
\begin{center}
|\input|\marg{path}
\end{center}
where \meta{path} is the file path you noted above. The input command will read 
the contents of |<edition>_preamble.tex| as if they were typed directly into 
your document.

Next, set the contents of the header\footnote{To see how this is done, look 
ahead to the section corresponding the preamble you are using.}, if applicable. 
Here is an example:

\Needspace*{11\baselineskip}
\begin{Verbatim}[fontsize=\small, frame=lines, framesep = 2ex,
    label = Sample Document, numbers = left, commandchars=+()]
\documentclass[letterpaper]{article}

\input{/home/yourname/Documents/homework_preamble.tex}
\coursetitle{Math Class}
\hwtitle{Homework \#1}
\myname{Joe Schmö}

\begin{document}
    +meta(Document content)
\end{document}
\end{Verbatim}
%% PACKAGE LOADING ERRORS
\subsection{Package Loading Errors}
Each preamble loads many packages. It is imperative that you follow these 
guidelines, else you might experience lots of errors when you compile your 
first document.

Firstly, there are some packages that may not be installed on your system. 
Refer to the \texttt{README} for an up-to-date list of these. Note that the 
\texttt{signchart} package, written by me, may need to be manually obtained 
from the CTAN.

Secondly, \emph{do not} change the order in which these packages are loaded by 
the preamble. Some packages depend on other packages, and some packages load 
other packages on their own. If there are specific options used by a certain 
package, then this package needs to be loaded with those options before another 
package loads it. All this is to say that if the packages are not loaded in the 
correct order, you will get likely get an ``\texttt{option clash}'' error when 
compiling your document, or some features may not work as expected.
%% USING MYLATEXFORMAT
\subsection{Using \texttt{mylatexformat}}
Creating a format file is an extremely effective method for speeding up the 
compilation time of your document. The following instructions assume you have 
basic knowledge of how to run commands in a terminal. If you have never done 
this, there are great
\href{http://pcsupport.about.com/od/commandlinereference/f/open-command-prompt.htm}
{guides} all
\href{http://www.macworld.co.uk/feature/mac-software/get-more-out-of-os-x-terminal-3608274/}
{over} the
\href{http://www.howtogeek.com/140679/beginner-geek-how-to-start-using-the-linux-terminal/}
{web}.

The preamble of any document consists of what we will call \emph{static} items 
and \emph{dynamic} items. The static items are the packages you load, the 
definitions you make --- things that rarely change for different documents. The 
dynamic items consist of things like |\title{}| or |\author{}| which are 
generally unique to the specific document in which you are working. We can 
create a format based on the static items and use this format in any document 
we wish. We are essentially ``precompiling'' the preamble to save time loading 
packages and defining commands every single time you compile a document.

Suppose you have a file called \texttt{mydoc.tex}, which may look something 
like this:
\Needspace*{18\baselineskip}
\begin{Verbatim}[fontsize=\small, frame=lines, framesep = 2ex,
    label = {mydoc.tex}, commandchars=+()]
\documentclass[letterpaper]{article}

% Static preamble items
\usepackage{+meta(package)}
...
\newcommand{+meta(command)}{+meta(definition)}
...

% Dynamic preamble items
\title{+meta(title)}
\author{+meta(author)}
...

\begin{document}
    +meta(Document content)
\end{document}
\end{Verbatim}
We have to tell \texttt{mylatexformat} where our static content ends, since 
that is the only part we wish to encapsulate into our format. We do this by 
adding the line |\csname endofdump\endcsname|, like so:
\Needspace*{19\baselineskip}
\begin{Verbatim}[fontsize=\small, frame=lines, framesep = 2ex,
    label = {mydoc.tex}, commandchars=+()]
\documentclass[letterpaper]{article}

% Static preamble items
\usepackage{+meta(package)}
...
\newcommand{+meta(command)}{+meta(definition)}
...

\csname endofdump\endcsname
% Dynamic preamble items
\title{+meta(title)}
\author{+meta(author)}
...

\begin{document}
    +meta(Document content)
\end{document}
\end{Verbatim}
Now we are ready to create a format. Open a terminal and navigate to the 
directory containing \texttt{mydoc.tex}. Then run the following command:
\begin{verbatim}
pdftex -ini -jobname="myformat" "&pdflatex" mylatexformat.ltx mydoc.tex
\end{verbatim}
The job name \texttt{myformat} will be the name of your new format. You can 
replace \texttt{\&pdflatex} with whichever program you will use to compile your 
document, such as \texttt{\&latex}, \texttt{\&xetex}, etc. After running this 
command, you will have a new file called \texttt{myformat.fmt}. Now, in your 
original document you can delete all of the contents before the 
\texttt{endofdump} line and replace them with a single line: |%&myformat|. Here 
is what your new document will look like:
\Needspace*{12\baselineskip}
\begin{Verbatim}[fontsize=\small, frame=lines, framesep = 2ex,
    label = {mydoc.tex}, commandchars=+()]
%&myformat
\csname endofdump\endcsname
% Dynamic preamble items
\title{+meta(title)}
\author{+meta(author)}
...

\begin{document}
    +meta(Document content)
\end{document}
\end{Verbatim}
A couple of notes:
\begin{enumerate}
\item The first line with your format name \emph{does} begin with a comment 
character, as shown.
\item You will no longer have the |\documentclass[]{}| command at the 
beginning. This is handled by your format.
\item The \texttt{endofdump} command must still be included, otherwise the 
dynamic preamble items will be skipped.
\end{enumerate}
This is all you need to start using formats. The next step is optional, but 
quite useful. The format file needs to be in the same directory as the document 
that is using it. If you wish to use the same format in several documents, 
there is a special directory in which you can put your format file that is 
searched automatically by the compiler, as long as you are using a standard TeX 
Live installation.
\begin{verbatim}
                        .../texmf/web2c/pdftex/
\end{verbatim}
With \texttt{myformat.fmt} copied into the above directory, you can use this 
format in any document.

For your convenience, format files have been created for the \textsc{Mottaz} 
preambles and are included in this distribution.
%%%%%%%%%%%%%%%%%%%%%%%%%%%%%%%%%%%%%%%%%%%%%%%%%%%%%%%%%%%%%%%%%%%%%%%%%%%%%%%%
%% HOMEWORK EDITION
%%%%%%%%%%%%%%%%%%%%%%%%%%%%%%%%%%%%%%%%%%%%%%%%%%%%%%%%%%%%%%%%%%%%%%%%%%%%%%%%
\section{\texttt{HOMEWORK} Edition}
This section describes the packages and macros associated with the file 
|homework_preamble.tex|. This preamble is designed to be used when typing 
mathematics homework solutions. To begin, let's start with a discussion about 
page layout.
%
%% HOMEWORK EDITION LAYOUT
\subsection{Page Layout}
The dimensions chosen for the page layout generally follow the recommendations
set out by typographers. Read the discussion
\href{http://tex.stackexchange.com/questions/71172/why-are-default-latex-%
margins-so-big}{here}. The author of the most up-voted answer mentions the
\texttt{tufte} document class, which was carefully created with significant
thought about proper typography practice. That class was intended for writing
books, and I think the margin size defined therein is too large to be
reasonable for writing mathematics homework assignments. However, I have
adjusted the layout of the page to partially conform with the proper typography
recommendations. For example, when using letter or A4 paper sizes, the text
width very nearly agrees with the recommended 60-75 characters per line. I feel
that this page layout is a nice compromise between the typographical guidelines
and the (relatively new) desire for roughly 1 inch margins.

Here is the page layout definition, using the 
\hyperref[pkg:geometry]{\texttt{geometry}} package.
\Needspace*{11\baselineskip}
\begin{Verbatim}[frame = lines, framesep = 2ex, fontsize = \small,
    label = Layout, numbers = left, firstnumber = 53]
\geometry{
    headheight = 1.3em,
    headsep = 1.3em,
    marginparsep = 1.5em,
    footnotesep = 3em,
    hdivide={0.15\paperwidth,0.62\paperwidth,*},
    vdivide={0.1\paperheight,0.78\paperheight,*},
    marginparwidth=0.18\paperwidth
}
\end{Verbatim}
%
%% THE PACKAGES
\subsection{Packages}
The following is a list of packages that are loaded into this preamble, along
with a description of what the package does and perhaps an example of its usage.
You may also click on each package to open the package documentation in your
internet browser.
%
\subsubsection{Page Layout Packages}
%
\DescribePackage%
{http://mirror.utexas.edu/ctan/macros/latex/required/tools/layout.pdf}%
{layout}
This package defines the command |\layout| which produces an overview of the
layout of the current document. The output of |\layout| for the present 
document can be found in Appendix \ref{apx:layout}.

\DescribePackage%
{http://mirrors.acm.jhu.edu/ctan/macros/latex/contrib/geometry/geometry.pdf}
{geometry}
This package provides a simple interface for defining and adjusting page 
dimensions. This is used to define the page layout in each preamble.
%
\subsubsection{Font and Typing Packages}
%
\DescribePackage%
{http://www.ctan.org/tex-archive/fonts/lm/}
{lmodern}
This package provides the Latin Modern font, an extension of the 
\href{http://www.ctan.org/tex-archive/fonts/cm/}{Computer Modern font} with 
extended glyph coverage and enhanced metrics.

\DescribePackage%
{http://www.ctan.org/pkg/fontenc}
{fontenc}
This package is a standard for changing the font encodings in the output PDF 
document. It is loaded as usual with the \texttt{T1} option, specifying the use 
of T1 encoding.

\DescribePackage%
{http://www.ctan.org/pkg/inputenc}
{inputenc}
This package, loaded with the \texttt{UTF-8} option, allows \LaTeX\ to read and 
understand special characters from your source document. For example, if you 
insert the character `ö' into your source document, it will give the expected 
output in your PDF.

\DescribePackage%
{http://mirrors.acm.jhu.edu/ctan/macros/latex/contrib/enumitem/enumitem.pdf}
{enumitem}
This package gives the user complete control over the customization of 
itemized/enumerated lists. This package is loaded with the \texttt{shortlabels} 
option so that the item labels can be customized easily by selecting 
\texttt{A}, \texttt{a}, \texttt{I}, \texttt{i}, or \texttt{1} and inserting 
whatever delimiters you want.
\Needspace*{7\baselineskip}
\begin{Verbatim}[frame = lines, framesep = 2ex, fontsize = \small,
    label = {Example using `enumitem'}]
\begin{enumerate}[\bfseries a.]
	\item First item
	\item Second item
	\item Third item
\end{enumerate}
\end{Verbatim}
Output:
\Needspace*{6\baselineskip}
\begin{shaded*}
\begin{enumerate}[\bfseries a., labelsep=*]
	\item First item
	\item Second item
	\item Third item
\end{enumerate}
\end{shaded*}
%
\subsubsection{Mathematics Packages}
%
\DescribePackage%
{http://mirrors.ibiblio.org/CTAN/macros/latex/contrib/mathtools/mathtools.pdf}
{mathtools}
This package is an extension to the standard
\href{http://mirror.utexas.edu/ctan/macros/latex/required/amsmath/amsmath.pdf}
{\texttt{amsmath}}
package. It loads the \texttt{amsmath} package, and also defines many useful 
and convenient macros and environments. For example, it provides the ``colon 
equals'' symbol $\coloneqq$ which I use frequently. It also allows for 
``cramped'' math, i.e. the \verb|\cramped{}| command changes $2^{2^2}$ to 
$\cramped{2^{2^2}}$. Refer to the documentation for a full description of its 
features.

\DescribePackage%
{http://ctan.mackichan.com/macros/latex/required/amscls/doc/amsthdoc.pdf}
{amsthm}
This package provides environments such as the \texttt{theorem} and 
\texttt{proof} environments. It also provides a method of creating new 
theorem-like environments by defining a |\theoremstyle| and creating a 
|\newtheorem|. See \hyperref[feat:theoremstyles]{below} for an example of this.

\DescribePackage%
{http://ctan.math.utah.edu/ctan/tex-archive/fonts/amsfonts/doc/amsfndoc.pdf}
{amssymb}
This package offers additional symbols to use in math mode. A complete list of 
these symbols can be found \href%
{http://milde.users.sourceforge.net/LUCR/Math/mathpackages/amssymb-symbols.pdf}
{here}.

\DescribePackage%
{http://ctan.mirrors.hoobly.com/fonts/doublestroke/dsdoc.pdf}
{dsfont}
This package provides double stroke letters to be used in math mode, such as
the natural numbers ($\N$), which is typeset with |\mathds{N}|. Several of
these are called with macros defined in the preamble, so we can achieve the 
following outputs:
\begin{center}
\begin{tabular}{c|c}
Command & Output \\ \hline
|\N| & $\N$ \\
|\Z| & $\Z$ \\
|\Q| & $\Q$ \\
|\R| & $\R$ \\
|\C| & $\C$ \\
|\H| & $\H$ \\
|\F| & $\F$ \\
\end{tabular}
\end{center}

\DescribePackage%
{http://ctan.mackichan.com/macros/latex/contrib/jknappen/mathrsfs.rme}
{mathrsfs}
This package provides script lettering to be used in math mode. For example, 
|\mathscr{ABC}| produces $\mathscr{ABC}$.

\DescribePackage%
{http://ctan.mackichan.com/macros/latex/contrib/cancel/cancel.pdf}
{cancel}
This package offers the \verb|\cancel{}| and \verb|\cancelto{}{}| commands (and 
a few others) which nicely draw a line through math expressions, and possibly 
draws an arrow to another value.
\Needspace*{5\baselineskip}
\begin{Verbatim}[frame = lines, framesep = 2ex, fontsize = \small,
    label = {Example using `cancel'}]
\[ \frac{x^2 - 4}{x+2}
= \frac{(x - 2)\cancelto{1}{(x + 2)}}{\cancel{x + 2}}
= x - 2. \]
\end{Verbatim}
Output:
\Needspace*{1in}
\begin{shaded*}
\[ \frac{x^2 - 4}{x+2}
= \frac{(x - 2)\cancelto{1}{(x + 2)}}{\cancel{x + 2}}
= x - 2. \]
\end{shaded*}

\DescribePackage%
{http://ctan.sharelatex.com/tex-archive/macros/latex/required/tools/array.pdf}
{array}
This package offers many helpful customization tools for \texttt{tabular} and 
\texttt{array} environments. This preamble uses the |\extrarowheight| parameter 
to add some height to each row in a \texttt{tabular} or \texttt{array} 
environment. Usually, if horizontal lines are used, then capital letters may 
touch the lines. This preamble defines |\extrarowheight| to be \texttt{1pt}, 
which gives enough buffer for letters to have space between horizontal lines 
and still look natural. Here is an example to show this difference:
\Needspace*{1in}
\begin{shaded*}
\begin{center}
\setlength{\extrarowheight}{0pt}
\begin{tabular}{|c|}
\hline
I FEEL A LITTLE SQUISHED \\ \hline
\end{tabular} \\[1em]
\setlength{\extrarowheight}{1pt}
\begin{tabular}{|c|}
\hline
THIS IS BETTER \\ \hline
\end{tabular}
\end{center}
\end{shaded*}

\DescribePackage%
{http://ctan.mirrors.hoobly.com/macros/latex/contrib/arydshln/arydshln-man.pdf}
{arydshln}
This package (read ``array dashed lines'') allows the user to use dashed lines 
in \texttt{tabular} and \texttt{array} environments. For vertical dashed lines, 
use the \texttt{:} option in place of \texttt{\symbol{"007C}}. For horizontal 
dashed lines, use |\hdashline| in place of |\hline|.
\Needspace*{6\baselineskip}
\begin{Verbatim}[frame = lines, framesep = 2ex, fontsize = \small,
    label = {Example using `arydshln'}]
\begin{tabular}{::l:c|r||}
Here is & an & example \\ \hdashline
Using & dashed & lines \\ \hline
\end{tabular}
\end{Verbatim}
Output:
\Needspace*{1in}
\begin{shaded*}
\begin{center}
\begin{tabular}{::l:c|r||}
Here is & an & example \\ \hdashline
Using & dashed & lines \\ \hline
\end{tabular}
\end{center}
\end{shaded*}

\DescribePackage%
{http://get-software.net/macros/latex/contrib/relsize/relsize-doc.pdf}
{relsize}
This package allows the user to change the size of math or text relative to 
what the current size is. For example, you can put an entire math expression 
inside of the |\mathlarger{}| command to increase the overall size for 
readability.
\Needspace*{5\baselineskip}
\begin{Verbatim}[frame = lines, framesep = 2ex, fontsize = \small,
    label = {Example using `relsize'}]
\[ \mathlarger{
	\frac{x + \frac{3}{2e^y}}{x^{2y-1}}
	} \]
\end{Verbatim}
Output:
\Needspace*{1in}
\begin{shaded*}
\[ \text{Before:}\quad \frac{x + \frac{3}{2e^y}}{x^{2y-1}} \qquad\qquad
\text{After:} \quad
\mathlarger{
	\frac{x + \frac{3}{2e^y}}{x^{2y-1}}
	} \]
\end{shaded*}

%\DescribePackage%
%{arg1}
%{arg2}
%
\subsubsection{Graphics and Plotting Packages}
%
\DescribePackage%
{http://ctan.math.utah.edu/ctan/tex-archive/macros/latex/contrib/xcolor/xcolor.pdf}
{xcolor}
This package gives the user access to a wider range of predefined colors, as 
well as several options for defining your own colors. This package is loaded 
with the \texttt{dvipsnames} option to load the 68 standard colors known to 
\texttt{dvips}:
\begin{center}
%\small \ttfamily
\begin{longtable}{|l|p{0.7cm}||l|p{0.7cm}||l|p{0.7cm}|}%
\hline
Name & Col & Name & Col & Name & Col \\ \hline \hline
Apricot & \myrule{Apricot} & Aquamarine & \myrule{Aquamarine} & Bittersweet & 
\myrule{Bittersweet} \\ \hline
Black & \myrule{Black} & Blue & \myrule{Blue} & BlueGreen & \myrule{BlueGreen} 
\\ \hline
BlueViolet & \myrule{BlueViolet} & BrickRed & \myrule{BrickRed} & Brown & 
\myrule{Brown} \\ \hline
BurntOrange & \myrule{BurntOrange} & CadetBlue & \myrule{CadetBlue} & 
CarnationPink & \myrule{CarnationPink} \\ \hline
Cerulean & \myrule{Cerulean} & CornflowerBlue & \myrule{CornflowerBlue} & Cyan 
& \myrule{Cyan} \\ \hline
Dandelion & \myrule{Dandelion} & DarkOrchid & \myrule{DarkOrchid} & Emerald & 
\myrule{Emerald} \\ \hline
ForestGreen & \myrule{ForestGreen} & Fuchsia & \myrule{Fuchsia} & Goldenrod & 
\myrule{Goldenrod} \\ \hline
Gray & \myrule{Gray} & Green & \myrule{Green} & GreenYellow & 
\myrule{GreenYellow} \\ \hline
JungleGreen & \myrule{JungleGreen} & Lavender & \myrule{Lavender} & LimeGreen & 
\myrule{LimeGreen} \\ \hline
Magenta & \myrule{Magenta} & Mahogany & \myrule{Mahogany} & Maroon & 
\myrule{Maroon} \\ \hline
Melon & \myrule{Melon} & MidnightBlue & \myrule{MidnightBlue} & Mulberry & 
\myrule{Mulberry} \\ \hline
NavyBlue & \myrule{NavyBlue} & OliveGreen & \myrule{OliveGreen} & Orange & 
\myrule{Orange} \\ \hline
OrangeRed & \myrule{OrangeRed} & Orchid & \myrule{Orchid} & Peach & 
\myrule{Peach} \\ \hline
Periwinkle & \myrule{Periwinkle} & PineGreen & \myrule{PineGreen} & Plum & 
\myrule{Plum} \\ \hline
ProcessBlue & \myrule{ProcessBlue} & Purple & \myrule{Purple} & RawSienna & 
\myrule{RawSienna} \\ \hline
Red & \myrule{Red} & RedOrange & \myrule{RedOrange} & RedViolet & 
\myrule{RedViolet} \\ \hline
Rhodamine & \myrule{Rhodamine} & RoyalBlue & \myrule{RoyalBlue} & RoyalPurple & 
\myrule{RoyalPurple} \\ \hline
RubineRed & \myrule{RubineRed} & Salmon & \myrule{Salmon} & SeaGreen & 
\myrule{SeaGreen} \\ \hline
Sepia & \myrule{Sepia} & SkyBlue & \myrule{SkyBlue} & SpringGreen & 
\myrule{SpringGreen} \\ \hline
Tan & \myrule{Tan} & TealBlue & \myrule{TealBlue} & Thistle & \myrule{Thistle} 
\\ \hline
Turquoise & \myrule{Turquoise} & Violet & \myrule{Violet} & VioletRed & 
\myrule{VioletRed} \\ \hline
White & \myrule{White} & WildStrawberry & \myrule{WildStrawberry} & Yellow & 
\myrule{Yellow} \\ \hline
YellowGreen & \myrule{YellowGreen} & YellowOrange & \myrule{YellowOrange} & & 
\\ \hline
\end{longtable}
\end{center}
To define a color, use |\definecolor{name}{model}{color-spec}|. For example, 
the background color to the output samples in this document was defined with
\begin{verbatim}
\definecolor{shadecolor}{RGB}{242,242,200}
\end{verbatim}
%
\subsubsection{Uncategorized Packages}
%
\end{document}